\nchapter{Zahlen}
\noindent Das Na’vi kennt ein \textit{oktales} Zahlensystem, also eines mit einer
Basis von acht, so wie eine sehr kleine Anzahl menschlicher Sprachen auch.
\footnote{Anscheinend das Ergebnis davon, da\ss{} nicht die Finger selber, sondern
deren Zwischenr\"aume gez\"ahlt werden.}
Anstatt Zahlen in der Form $(m \times 10) + n$ (wie beispielsweise $(4 \times 10) + 2 = 42_{10}$,
\D{zweiundvierzig}), zu berechnen, wird in der Form $(m \times 8) + n$ (wie beispielsweise
$(5 \times 8) + 2 = 52_8$, \N{mrrvomun} gerechnet, was $42_{10}$ entspricht).

\section{Kardinalzahlen}
\index{Zahlen!Kardinal-}

\subsection{Die Einerstelle} Die eigenst\"andigen Formen der Zahlenwerte von eins
bis acht sind:

\begin{center}
\begin{tabular}{ll}
1 & \N{’aw} \\
2 & \N{\ACC{mu}ne} \\
3 & \N{pxey} \\
4 & \N{ts\`ing} \\
\end{tabular}
\hskip 3em
\begin{tabular}{ll}
5 & \N{mrr} \\
6 & \N{\ACC{pu}kap} \\
7 & \N{\ACC{ki}nä} \\
8 & \N{vol} \\
\end{tabular}
\end{center}

\subsection{Potenzen von acht} Anstatt Zehnern hat das Na’vi Achter:

\begin{center}
\begin{tabular}{ll}
8 (1 $\times$ 8) & \N{vol} \\
16 (2 $\times$ 8) & \N{\ACC{me}vol} \\
24 (3 $\times$ 8) & \N{\ACC{pxe}vol} \\
32 (4 $\times$ 8) & \N{\ACC{ts\`i}vol} \\
\end{tabular}
\hskip 3em
\begin{tabular}{ll}
40 (5 $\times$ 8) & \N{\ACC{mrr}vol} \\
48 (6 $\times$ 8) & \N{\ACC{pu}vol} \\
56 (7 $\times$ 8) & \N{\ACC{ki}vol} \\
64 (8 $\times$ 8) & \N{zam} \\
\end{tabular}
\end{center}

\newpage
\subsection{Zusammengesetzte Formen} Werden sie mit Achterpotenzen kombiniert, so
werden die Zahlw\"orter einsilbig und lenisieren, wenn m\"oglich:\label{numbers:dependent}
\index{Lenition!Zahlen}

\begin{center}
\begin{tabular}{ll}
1 & \N{(l)-aw} \\
2 & \N{-mun} \\
3 & \N{-pey} \\
4 & \N{-s\`ing} \\
\end{tabular}
\hskip 3em
\begin{tabular}{ll}
5 & \N{-mrr} \\
6 & \N{-fu} \\
7 & \N{-hin} \\
\end{tabular}
\end{center}

\subsubsection{} Alle zusammengesetzten Formen mit Ausnahme von "`eins"', \N{(l)-aw},
lassen das End-\N{-l} der Achterwerte fallen.

\subsubsection{} Die angeh\"angten Teile bekommen die Betonung. Kombiniert mit
\N{vol}, also \D{acht}:

\begin{center}
\begin{tabular}{ll}
9 (1$\times$8 $+$ 1) & \N{vo\ACC{law}} \\
10 (1$\times$8 $+$ 2) & \N{vo\ACC{mun}} \\
11 (1$\times$8 $+$ 3) & \N{vo\ACC{pey}} \\
12 (1$\times$8 $+$ 4) & \N{vo\ACC{s\`ing}} \\
\end{tabular}
\hskip 3em
\begin{tabular}{ll}
13 (1$\times$8 $+$ 5) & \N{vo\ACC{mrr}} \\
14 (1$\times$8 $+$ 6) & \N{vo\ACC{fu}} \\
15 (1$\times$8 $+$ 7) & \N{vo\ACC{hin}} \\
16 (2$\times$8 $+$ 0) & \N{\ACC{me}vol} \\
\end{tabular}
\end{center}

\noindent Dieses Muster setzt sich fort mit \N{\ACC{me}vol}:
\N{mevo\ACC{law}}, \N{mevo\ACC{mun}}, \N{mevo\ACC{pey}}, usw.

\section{Ordinalzahlen}

\subsection{Suffix -ve} Die Ordinalzahlen werden gebildet, indem der Suffix \N{-ve}
angeh\"angt wird, was die Betonung des Wortes nicht \"andert, obwohl es ein paar
Wortst\"amme ver\"andert.\index{-ve@\textbf{-ve}}\index{Zahlen!Ordinal-}

\begin{center}
\begin{tabular}{rll}
Ordinalzahl & Eigenst\"andig & Zusammengesetzt \\
\hline
first & \N{\ACC{’aw}ve} & \N{(l)-\ACC{aw}ve} \\
second & \N{\ACC{mu}ve} & \N{-\ACC{mu}ve} \\
third & \N{\ACC{pxey}ve} & \N{-\ACC{pey}ve} \\
fourth & \N{\ACC{ts\`i}ve} & \N{-\ACC{s\`i}ve} \\
\end{tabular}
\hskip2em
\begin{tabular}{rll}
\\
fifth & \N{\ACC{mrr}ve} & \N{-\ACC{mrr}ve} \\
sixth & \N{\ACC{pu}ve} & \N{-\ACC{fu}ve} \\
seventh & \N{\ACC{ki}ve} & \N{-\ACC{hi}ve} \\
eighth & \N{\ACC{vol}ve} & \N{-volve} \\
\end{tabular}
\end{center}

\subsubsection{} \QUAESTIO{Kann es beliebig mit \N{nì-} kombiniert werden?}


\section{Br\"uche}
\index{Zahlen!Br\"uche}\index{Br\"uche}

\subsection{-Px\`i} Mit Ausnahme von \D{halb} und \D{Drittel}, f\"ur die es eigenst\"andige
lexikalische Formen gibt, indem statt des \N{-ve} ein \N{-px\`i} angeh\"angt wird. Beachten
Sie hier bitte die Verschiebung der Betonung:\index{-pxiì@\textbf{-px\`i}}

\begin{center}
\begin{tabular}{rl}
H\"alfte & \N{mawl} \\
Drittel & \N{pan} \\
Viertel & \N{ts\`i\ACC{px\`i}} \\
F\"unftel & \N{mrr\ACC{px\`i}} \\
\end{tabular}
\hskip 2em
\begin{tabular}{rl}
Sechstel & \N{pu\ACC{px\`i}} \\
Siebtel & \N{ki\ACC{px\`i}} \\
Achtel & \N{vo\ACC{px\`i}} \\
\\
\end{tabular}
\end{center}

\subsubsection{Wortgruppe} Beachten Sie bitte, da\ss{} es sich bei den Br\"uchen im Gegensatz
zu den Ordinalzahlen um Substantive und nicht um Adjektive handelt.
(Siehe auch \horenref{syn:partitive-gen} f\"ur den Satzbau).

\subsection{Z\"ahler} Um h\"oherwertige Br\"uche zu erzeugen, verwenden Sie eine attributive
Kardinalzahl mit einem Bruch: \N{munea mrrpx\`i}, \D{zwei F\"unftel}.

\subsubsection{Zwei Drittel} Der Bruch \D{zwei Drittel} verf\"ugt \"uber eine spezielle
Schreibweise, \N{mefan}, dem Dual von \N{pan}.\index{mefan@\textbf{mefan}}


\section{Sonderformen}

\subsection{Alo} Das Wort \N{\ACC{a}lo}, \D{-mal, -fach} kann an Zahlen angeh\"angt werden,
um H\"aufigkeitsadverben zu bilden. Vier davon sind zusammengesetzte W\"orter:
\N{\ACC{’aw}lo} \D{einmal, einfach}, \N{\ACC{me}lo} \D{zweimal, doppelt}, \N{\ACC{pxe}lo}
\D{dreimal, dreifach} and \N{\ACC{fra}lo} \D{jedesmal}.
Alle anderen werden als normale attributive Adjektive gebildet: \Npawl{\uwave{alo
amrr} poan polawm} \D{Er fragte \uwave{f\"unfmal}}. \index{melo@\textbf{melo}}
\index{'awlo@\textbf{’awlo}}\index{alo@\textbf{alo}}\index{fralo@\textbf{fralo}}
\index{pxelo@\textbf{pxelo}}

\subsection{-lie} Das Wort \N{’aw\ACC{li}e} bezieht sich auf ein einziges Ereignis
in der Vergangenheit.\index{-lie@\textbf{-lie}}\index{'awlie@\textbf{’awlie}}

\subsection{Fremde Ziffern} Werden menschliche Zahlen zitiert, dann werden im Na’vi
die aus dem Englischen \"ubernommenen W\"orter \N{’eyt} f\"ur \D{(eng. eight) acht} und
\N{nayn} f\"ur \D{(eng. nine) neun} verwendet. Diese werden nicht f\"urs Z\"ahlen eingesetzt,
sondern um Telefonnummern, o. \"a. wiederzugeben.\index{'eyt@\textbf{’eyt}}\index{nayn@\textbf{nayn}}

\subsubsection{} \N{Kew} bedeutet \D{Null}. \QUAESTIO{Die aktuelle Dokumentation gibt
keinen Aufschlu\ss{} dar\"uber, ob den Na’vi dieses Konzept schon bekannt gewesen ist
oder erst von den Menschen \"ubernommen wurde.}\index{kew@\textbf{kew}}
