\nchapter{Pragmatik und Diskurs}

\noindent In den vorigen Kapiteln habe ich Laute, W\"orter und S\"atze im Na’vi
diskutiert. Ein gro\ss{}er Teil dieser Diskussion hat die Form von Regeln angenommen.
Dieses Kapitel behandelt Sprache eine Stufe h\"oher als selbst der Satz ---
Unterhaltung, Erz\"ahlung und der Kontext, in dem die Sprache stattfindet, was
Linguisten unter der Bezeichnung Pragmatik zusammenfassen. Einfache Regeln wird man
hier seltener finden, von daher weist diese Diskussion eine etwas andere Struktur
auf.

\section{Reihenfolge der Satzglieder}

\subsection{Freie Wortreihenfolge}  Das Na’vi wurde als Sprache beschrieben, deren
Wortreihenfolge nicht fest vorgegeben ist. Dies ist etwas irref\"uhrend, da diese
Aussage eine spezifische Bedeutung f\"ur einen Linguisten hat. Stattdessen verf\"ugt
das Na’vi \"uber eine freie Reihenfolge der Satzglieder\footnote{Ein
\textit{Satzglied} ist eine etwas gr\"o\ss{}ere Einheit als ein Wort, die aber
kleiner als ein Satz ist. Ein Satzglied ist eine Gruppe W\"orter, die eine
bestimmte grammatikalische Einheit bilden. So stellt beispielsweise in dem Satz "`Der
gro\ss{}e, b\"ose Wolf fra\ss{} Rotk\"appchans Gro\ss{}mutter"' der Teil "`Der gro\ss{}e,
b\"ose Wolf"' das Subjekt des Satzes dar, das Verb "`fra\ss{}"' steht f\"ur sich, und
"`Rotk\"appchens Gro\ss{}mutter"' bildet das direkte (oder Akkusativ-)Objekt. Manchmal
kann ein Satzglied aus einem einzigen Wort bestehen, ("`Er fra\ss{} sie"' --- jedes Wort 
ist ein Satzglied) und manchmal k\"onnen sie etwas erheblich Komplexeres sein.}.
Innerhalb eines Satzgliedes kann die Wortreihenfolge mehr oder minder fest vorgegeben
sein. So kann man einen Teil eines Satzgliedes nicht einfach in einem anderen unterbringen.
So kann man beispielsweise in \N{Ayoel tarmaron tsawla yerikit}, \D{Wir jagten einen
gro\ss{}en Hexapeden}, nicht einfach das Akkusativobjekt \N{tsawla yerikit} aufbrechen,
um etwas wie *\N{tarmaron tsawla ayoel yerikit} oder *\N{ayoel tsawla tarmaron yerikit}
zu erzeugen.
\index{Wortreihenfolge}

\subsubsection{} In komplexen Satzgliedern kann ein Genitiv durch einen Relativsatz von
dem Substantiv, auf das es sich bezieht, abgetrennt sein: \N{Ngey\"a teri faytele a
ays\"anumeri}, \D{Deine Anweisungen bez\"uglich dieser Angelegenheiten}.

\subsection{SOV, SVO, VSO} Viele menschliche Sprachen k\"onnen bequem anhand der
Standardreihenfolge ihrer Satzglieder Subjekt, Verb (oder auch Pr\"adikat) und (direktes)
Objekt, normalerweise als S, V und O abgek\"urzt, einordnen.
Englisch ist in den meisten F\"allen eine SVO-Sprache, genau wie das Deutsche\footnote{In
Nebens\"atzen wandert das Verb (nicht zwingend notwendigerweise der Rest des Pr\"adikats!)
jedoch nach hinten, wodurch sich SOV als Reihenfolge ergibt.}, und das Japanische SOV.
Sprachen mit freier Wortreihenfolge sind schwieriger in dieses System einzuordnen, obwohl
sie einige von ihnen bemerkenswerte Tendenzen aufweisen. Wenn man sich einmal Frommers
Na’vi anschaut und nur S\"atze betrachtet, die alle drei Satzglieder aufweisen, so
kristallisieren sich sehr schnell SVO, SOV und VSO als Hauptreihenfolgen heraus, mit einer
sehr schwachen Pr\"aferenz f\"ur VSO. Andere Reihenfolgen wie OVS und OSV sind viel
seltener.\footnote{Dies basiert auf zwei gr\"o\ss{}eren Schiftst\"ucken mit zusammenh\"angendem
Na’vi, die Frommer geschrieben hatte, einmal sein erster Blogeintrag, und zum Zweiten seine
Nachricht auf MaSempul.org.
\begin{center}
\begin{tabular}{llll}
\N{Reihenfolge} & \N{Blog} & \N{Ma Sempul} & Gesamt \\
\hline
SVO & 2 & 3 & 5\\
SOV & 4 & 1 & 5\\
VSO & 5 & 2 & 7\\
OVS & 0 & 1 & 1\\
OSV & 0 & 2 & 2\\
\end{tabular}
\end{center}}

\subsection{Effekte der Wortreihenfolge}\index{Wortreihenfolge!Effekte} Eine
\"Anderung der Wortreihenfolge kann manchmal Ver\"anderungen in der Grammatik
verursachen.\label{pragma:woe}

\subsubsection{}\label{pragma:word-order-effects:modals}
Wenn ein Satz so angeordnet ist, da\ss{} das Modalverb und das von ihm kontrollierte
transitive Verb nebeneinander stehen, genauso wie das Subjekt und das direkte
Objekt, kann die Kombination aus Modalverb und Verb als ein einziges transitives
Verb aufgefa\ss{}t werden. Beispielsweise weist \Npawl{Oe teylut new yivom},
\D{Ich m\"ochte Teylu essen} die erwartete, richtige Anwendung der F\"alle auf, mit
dem Subjekt im Nominativ und dem direkten Objekt im Patiens (\horenref{syn:modals}).
Allerdings kann das Subjekt in einigen Wortstellungen im Agens stehen. In absteigender
Reihenfolge der Akzeptabilit\"at:

\begin{center}
\begin{tabular}{lr}
\N{\uwave{Oel} teylut new yivom.} & allgemein akzeptabel\footnotemark \\
\N{Teylut \uwave{oel} new yivom.} & etwa zu 50\% akzeptabel \\
\N{New yivom teylut \uwave{oel}.} & etwa zu 30\% akzeptabel \\
*\N{New yivom oel teylut.} & vollst\"andig inakzeptabel
\end{tabular}
\end{center}
\footnotetext[\value{footnote}]{Nach Frommers Blog, "`...in allen au\ss{}er den
h\"ochstformellen Situationen..."'}

\subsection{Hervorhebung} Da in Sprachen mit freier Wortreihenfolge diese nicht f\"ur die
Satzkonstruktion relevant ist, kann man sie beliebig einsetzen, um andere Dinge anzuzeigen,
wie Stil, Hervorhebung und Fokus. Die einzige gesicherte Aussage Frommers in Bezug auf die
Wortreihenfolge ist diese: "`Das Ende eines Satzes ist die Stelle, an der der `Schlag'
ansetzt."'
Wir k\"onnen das so auffassen, da\ss{}, wenn man ein Satzglied hervorheben m\"ochte, man
es ans Ende des Satzes stellt. Beachten Sie besonders, wie Frommer diesen Satz
\"ubersetzt hat:

\begin{quotation}
\noindent\Npawl{Fkxilet a tsawfa poe ioi säpalmi ngolop \uwave{Va'rul}}. \\
\indent\D{\uwave{Va'ru} ist diejenige, die die Halskette, die sie tr\"agt, hergestellt hat}.
\end{quotation}
% http://naviteri.org/2011/08/new-vocabulary-clothing/

\subsection{Der Passiv im Deutschen}\index{Passiv!und Wortreihenfolge}
Der Passiv erlaubt es uns, den "`Verursachten"' einer Handlung hervorzuheben, indem er
an den Anfang des Satzes gestellt wird. Wenn wir sagen: "`Die Nonne wurde von dem Auto
\"uberfahren"', so sticht die Nonne am deutlichsten hervor, der Art des Autos wird
weniger Beachtung geschenkt.\footnote{Im Deutschen kann man den Verursacher ganz
fallenlassen: "`Die Nonne wurde \"uberfahren"'.} Das Na’vi verf\"ugt nicht \"uber den
Passiv, aber Frommer hatte vorgeschlagen, da\ss{} die Wortreihenfolge OSV eingesetzt
werden kann, um den gleichen Effekt hervorzurufen (aber siehe auch \N{fko},
\horenref{syn:prn:fko}).


% \QUAESTIO{Thetic vs. categorical statements.}


\section{Themenbezogener Kommentar}
\label{pragma:topic-comment}\index{Fall!thematisch}

\noindent Die Konstruktion mit themenbezogenem Kommentar\footnote{in Ermangelung
einer brauchbareren \"Ubersetzung aus dem Englischen habe ich es als "`themenbezogener
Kommentar"' \"ubersetzt.} ist von der Konzeption her unkompliziert: Das "`Thema"'
gibt an, worauf sich der Rest des Satzes bezieht, und der Kommentar t\"atigt eine
Aussage bezogen auf das Thema. Viele menschliche Sprachen organisieren einen
Diskurs stark um Strukturen themenbezogener Kommentare. Deutsch geh\"ort leider
nur sehr bedingt dazu.
Dies kann es schwierig gestalten, eine ordentliche \"Ubersetzung solcher Strukturen
zu liefern, die sowohl die Bedeutung des Originals korrekt wiedergeben als auch die
Struktur in einer Diskussion verdeutlicht.
Daher werde ich in diesem Abschnitt oft auf Pr\"apositionalkonstruktionen mit "`bezogen
auf"' respektive "`betreffend"' f\"ur alle Beispiele verwenden, aber das ist lediglich
eine sperrige Hilfskonstruktion, die nur der Verdeutlichung dient.

\subsection{Thematischer Fall} Im Na’vi k\"onnen nur Substantive, Nominalphrasen und
Pronomen Themen sein. Diese werden mit dem thematischen Fall versehen (\N{-ri, -\`iri}).
Komplexere Themen k\"onnen mit Nominalphrasen gebildet werden (\horenref{syn:clause-nom}).

\subsection{Rolle dieses Falls} Der themenbezogene Fall kann f\"ur diejenigen, die damit
nicht vertraut sind, besonders verwirrend sein, gerade weil beinahe jede syntaktische
Rolle aus dem Satz herausgezogen werden kann, um als Thema zu fungieren. Ein idiomatischer
Gebrauch ist der unver\"au\ss{}erliche Besitz (\horenref{syn:topical:poss}). Man kann
den thematischen Fall \"uberall verwenden, wo das Deutsche ein Akkusativobjekt verwendet.

\begin{quotation}
\noindent\Npawl{Fayupxare layu aysng\"a’iyufpi, fte \uwave{l\`i’fyari awngey\"a} fo
ts\`iyevun n\`iftue n\`iltsans\`i nivume.}

\medskip
\noindent\D{Diese Nachrichten sind f\"ur Anf\"anger, so da\ss{} sie \uwave{unsere Sprache}
einfach und gut lernen k\"onnen}.
\end{quotation}

\noindent Die Beziehung des Themas mu\ss{} ebenfalls nicht eine bestimmte syntaktische
Rolle repr\"asentieren:

\begin{quotation}
\noindent\Npawl{Ma oey\"a eylan, \uwave{fays\"anumviri} rutxe f\`i’ut tslivam: \dots}\\
\indent\D{Meine Freunde, \uwave{bez\"uglich dieser Lektionen} habt bitte Verst\"andnis f\"ur \dots}

\medskip
\noindent\Npawl{\uwave{Ayngey\"a s\`ipawm\`iri} kop fmayi f\`itsenge tiv\`ing s\`i’eyngit}. \\
\indent\D{\uwave{Was eure Fragen betrifft,} so werde (ich) versuchen Antworten (darauf) zu geben}.
\end{quotation}

\subsubsection{} Ein Thema kann einen komplexen Satz einleiten und steht dann sogar vor der
einleitenden Konjunktion:

\begin{quotation}
\noindent\Npawl{\uwave{Fori} mawkrra fa renten ioi s\"apoli holum.}\\
\indent\D{Nachdem sie ihre Schutzbrillen aufgesetzt hatten, brachen sie auf}.
\end{quotation}
% http://naviteri.org/2011/08/new-vocabulary-clothing/

\subsubsection{} Der themenbezogene Fall kann ebenso f\"ur mehrfache Kommentare
verwendet werden:

\begin{quotation}
\noindent\Npawl{\uwave{Poeri} unilt\`irantokxit tarmok a krr, lam stum n\`iayfo,
sl\"a lu ’a’awa t\`iketeng --- natkenong, \uwave{tsyokx\`iri} ke lu zekw\"a ats\`ing
ki amrr.}

\medskip
\noindent\D{\uwave{Was sie betrifft}, so war \uwave{sie} beinahe wie sie, als
\uwave{sie} ihren Avatar bewohnte, aber es gab ein paar Unterschiede ---
beispielsweise \uwave{was ihre Hand betrifft}: Diese hatte nicht nur vier, sondern
f\"unf Finger}.
\end{quotation}


\subsection{Den themenbezogenen Fall verwenden} Jede menschliche Sprache verf\"ugt
\"uber eigene Regeln und Tendenzen, wann der thematische Fall eingesetzt werden
sollte. Zu diesem Zeitpunkt ist es etwas schwierig, Regeln daf\"ur abzuleiten, aber
ein paar Tendenzen k\"onnen von dem, was wir bisher gesehen haben, abgeleitet
werden. Zuerst einmal hat Frommer den themenbezogenen Fall bisher bei Weitem nicht
so oft eingesetzt, wie es im Chinesischen oder Japanischen (beides Sprachen, die
ausgiebig davon Gebrauch machen), \"ublich ist, und zweitens benutzt Frommer diesen
Fall nicht, um neue Themen einzuf\"uhren, sondern die Themen beziehen sich eher
auf aktuelle Angelegenheiten oder auf solche, die sich aus dem Gespr\"ach ableiten
lassen.

Das Deutsche verwendet den bestimmten Artikel, \D{der, die, das}, um Informationen,
die bereits in eine Rede eingabracht wurden, sowie Informationen, die man anhand
einer Unterhaltung annehmen oder aus ihr herleiten kann, anzuzeigen.
Beispielsweise, wenn ich sage: "`Ich wollte \D{Avatar} sehen, aber die Schlange
war zu lang"', so kann ich den bestimmten Artikel hier verwenden, nicht weil wir
uns \"uber Menschenschlangen unterhalten haben, sondern weil wir es gewohnt sind
anzustehen, wenn wir einen popul\"aren Film sehen wollen. In einem Kommentar auf
einen k\"urzlich get\"atigten Blogeintrag\footnote{\href{http://naviteri.org/2010/08/20/}{A
Na'vi Alphabet}, August 20, 2010} schreibt Frommer:\index{Fall!thematisch!bestimmt}

\begin{quote}Aber wenn die Nachricht unbestimmt ist, l\"a\ss{}t der thematische Fall
sich nicht so gut anwenden, da Themen im Regelfalle bestimmter Natur sind. Daher kann
\N{’Upxareri ngaru pamrel soli trram} sicherlich \D{Ich habe Dir \textbf{die} Nachricht
gestern geschrieben} bedeuten. Kann es aber auch \D{Ich habe Dir \textbf{eine}
Nachricht geschrieben} bedeuten? Da es im Na’vi per se keine Artikel gibt und
Substantive sowohl bestimmt als auch unbestimmt sein k\"onnen, gehe ich davon aus,
da\ss{} dem so sein k\"onnte. Aber irgendetwas geht mir dabei gegen den Strich.
\end{quote}

\noindent Es scheint am besten zu sein, unbestimmte Themenbez\"uge derzeit zu vermeiden.


\section{Sprachebene}

\subsection{Formelle Sprachebene} Das Na’vi verf\"ugt \"uber zwei M\"oglichkeiten, um
zeremonielle oder formelle Sprache anzuzeigen: Mit speziellen Pronomen
(\horenref{morph:hon-pron}) oder mit dem Verbinfix \N{\INF{uy}}
(\horenref{morph:verb:2nd-pos}).\index{Sprachebene!formell}

\subsubsection{} Die formellen Pronomen k\"onnen in dichter Folge eingesetzt werden:
\N{S\`ifmetokit emzola’u \uwave{ohel}. \"Atx\"ale si tsn\`i livu \uwave{oheru}
Uniltaron} \D{Ich habe alle Pr\"ufungen bestanden, Ich erbitte h\"oflich die
Traumjagd}.

\subsubsection{} Wie mit Zeitform- und Aspektmarker auch ist es nicht notwendig, da\ss{}
der Infix \N{\ACC{uy}} wiederholt wird, sowie der formelle Kontext einmal etabliert
ist.

\subsubsection{} Feierlichkeit und Ernsthaftigkeit einer Aussage k\"onnen durch den
Einsatz sowohl des Markers f\"ur das Pronomen als auch dem f\"ur das Verb angezeigt
werden: \Npawl{Faysulf\"atu\"a t\`ikangkem \uwave{oheru} meuia \uwave{luyu} n\`ingay},
\D{Die ARbeit f\"ur diese Experten ist wahrlich eine Ehre f\"ur mich}.

\subsection{Umgangssprachliche Sprachebene} Die umgangssprachliche Ebene zeigt sich
\"uberwiegend durch vereinfachte Grammatik und verk\"urzte Ausdr\"ucke.
\index{Sprachebene!umgangssprachlich}

\subsubsection{} Verben der Erkenntnis k\"onnen Nebens\"atze ohne Konjunktion einleiten.

\begin{quotation}
\noindent\D{Ich glaube, es war ein Fehler, da\ss{} er gegangen ist}. \\
\noindent\Npawl{\uwave{Sp\"angaw oel futa} fwa po kol\"a lu kxeyey.}\\
\noindent Umgangssprachlich: \Npawl{\uwave{Spaw oe}, fwa po kol\"a l\"angu kxeyey.}
\end{quotation}


\subsection{Verk\"urzte Sprachebene} Mit einem milit\"arischen Hintergrund k\"onnen
bestimmte Eigenschaften der Grammatik ver\"andert oder ausgelassen werden.
\index{Sprachebene!milit\"arisch}\index{Sprachebene!verk\"urzt}

\subsubsection{} In Nominalphrasen\"au\ss{}erungen k\"onnen Partizipien ohne den
attributiven Affix \N{-a-} verwendet werden (\horenref{syn:part:attr}):
\N{t\`ikan tawnatep}, \D{Ziel verloren} (aus dem Videospiel).

\subsubsection{} Der Genitiv mancher Pronomen verlieren das letzte \N{-\"a}. Siehe
auch \horenref{morph:pron:gen-clipped}. Dies kann in nichtmilit\"arischen Situationen,
beispielsweise unter Freunden oder nahen Verwandten, auch beil\"aufig verwendet werden.


\subsection{Poetische Sprachebene}

\subsubsection{} In Prosa erscheint der thematische Fall an erster Stelle eines
Gliedsatzes oder direkt nach der Konjunktion (\horenref{syn:topical:word-order}).
In Poesie kann er weiter in die Struktur des Gliedsatzes hinein wandern:
\Npawl{\uwave{Pxan} livu txo n\`i’aw oe \uwave{ngari} / tsakrr nga
Na’viru yomt\`iy\`ing}, \D{Nur wenn ich \uwave{Deiner w\"urdig bin}, wirst Du das
Volk versorgen}.

\subsubsection{} Steht in normaler Prosa eine Adposition vor dem Substantiv oder der
Nominalphrase, so m\"ussen etwaige Genitive ebenfalls hinter der Adposition stehen,
wie in \N{fa oey\"a tsyokx} oder \N{fa tsyokx oey\"a}, \D{mit meiner Hand}. In der
Poesie kann der Genitiv auch vor der Adposition stehen: \N{oey\"a fa tsyokx}. 
\LNWiki{17/3/2012}{http://wiki.learnnavi.org/index.php/Canon/2012/January-June\%23A_poetic_license_and_a_note_on_adposition_position}
