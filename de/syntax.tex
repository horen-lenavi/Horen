\nchapter{Satzbau}

\section{Transitivit\"at und Ergativit\"at}

\subsection{Transitivit\"at} Das Na’vi kennzeichnet die Subjekte transitiver und
intransitiver Verben unterschiedlich. Einen Satz auf Na’vi sprechen zu k\"onnen,
verlangt ein Verst\"andnis von Transitivit\"at. Das bedeutet, da\ss{} im Normalfalle
ein wesentlich fr\"uheres und tiefers Verst\"andnis von Transitivit\"at f\"ur das Na’vi
vonn\"oten ist als f\"ur viele menschliche Sprachen.\footnote{Im Gegensatz zu Sprechern
des Englischen haben es Sprecher des Deutschen hier einfacher, da sich ein transitives
Verb anhand des notwendigen Akkusativobjektes (das direkte Objekt) erkennen l\"a\ss{}t.
Im Englischen ist die Unterscheidung, ob ein Verb transitiv oder intransitiv ist,
aufgrund fehlender eindeutiger F\"alle ungleich schwieriger, was zus\"atzlich dadurch
verkompliziert wird, da\ss{} es sehr oft nicht das Verb ist, das transitiv oder
intransitiv ist, sondern die gesamte Satzkonstruktion. So ist "`I move"' (Ich bewege
mich) intransitiv und "`I move the car"' (Ich bewege das Auto) transitiv (beachten
Sie bitte, da\ss{} die \"Ubersetzungen ins Deutsche beide transitiv sind, da sie beide
\"uber ein Akkusativobjekt verf\"ugen. Im ersten Fall ist es der R\"uckbezug, der die
Transitivit\"atsbedingung erf\"ullt), und nur die Anwesenheit des direkten Objektes
sorgt f\"ur die Transitivit\"at des Verbs. Im Na’vi ist es ebenfalls am besten, sich
Transitivit\"at als satz- denn als wortbasiertes Ph\"anomen vorzustellen.

Es gibt zwei schnelle M\"oglichkeiten, um im Englischen auf Transitivit\"at zu testen:
Die erste M\"oglichkeit ist zu \"uberpr\"ufen, ob dem Verb direkt ein Substantiv folgt.
Das Verb in "`I see the moon"' ist transitiv, das Verb in "`He complains constantly"'
ist es nicht. Die zweite M\"oglichkeit ist festzustellen, ob sich der Satz sinnvoll
in eine Passivkonstruktion umstellen l\"a\ss{}t. "`The moon is seen"' ist eine
g\"ultige Passivkonstruktion, "`Constantly is complained"' ist hingegen Unsinn.
Im Deutschen ist hier lediglich die Suche nach dem Akkusativobjekt vonn\"oten, welches
sich anhand eines etwaigen vorangestellten Artikels und der Fallendung erkennen
l\"a\ss{}t.}\index{Transitivit\"at}

\subsubsection{} Viele zusammengesetzte Verben werden durch die Kombination eines
undeklinierten Substantivs oder Adjektivs mit dem Verbstamm \N{si}, \D{tun, machen},
gebildet, welcher einzig in diesen Konstruktionen Anwendung findet. \N{irayo si},
\D{danken} und \N{kavuk si}, \D{verraten}.
Diese Verben sind immer intransitiv und erhalten im Bedarfsfalle ein Dativobjekt.
(\horenref{syn:case:dative-si}).

\subsubsection{} Reflexive Verben, die mit dem \N{INF{\"ap}}-Infix gebildet werden,
sind immer intransitiv und Verben mit dem \N{\INF{eyk}}-Infix immer transitiv.

\subsection{Dreigliedrigkeit} Das Na’vi zeichnet Substantive und Pronomen unterschiedlich
aus, je nachdem ob sie das Subjekt eines intransitiven Verbs, das Subjekt oder das direkte
Objekt eines transitiven Verbs sind (\horenref{syn:cases}).

\subsubsection{} Obwohl das Konzept des Subjektes aus dem Deutschen im Na’vi bezogen auf
die Transitivit\"at des Verbs zweigeteilt ist, gilt diese Teilung nicht f\"ur Partizipien.
Es gibt eine adjektivische Objektkomponente (das Partizip Passiv) sowie eine adjektivische
Subjektkomponente (das Partizip Aktiv), wobei letztere sowohl mit dem Nominativ als auch
mit dem Agens Verwendung findet (\horenref{morph:pre-first}).

\subsubsection{} \QUAESTIO{Das Na’vi hat im Praktischen einen geteilten Ergativ. In
zusammenh\"angender Rede kann das Subjektpronomen entfallen, wenn es unver\"andert
bleibt. Das Subjekt kann dann entweder der Nominativ oder der Agens sein.
Siehe auch den Abschnitt \"uber \textit{praktische Anwendungen}.}


\section{Nominalphrasen und Adjektive}

\subsection{Anzahl} \QUAESTIO{Sind die Kollektiva f\"ur den Dual und dem Trial gegen\"uber
dem Plural distributiv oder zwingend erforderlich?}

\subsubsection{} Wird ein Substantiv zusammen mit einem Zahlwort verwendet, so \"andert
sich dessen Numerus nicht: \N{mrra z\`is\`it}, \D{f\"unf Jahre}.
\index{Plural!kein Numerus mit Zahlw\"ortern}
\LNWiki{18/6/2010}{http://wiki.learnnavi.org/index.php/Canon/2010/March-June\%23Numbers_take_nouns_in_the_SINGULAR.}

\subsubsection{} Die quantitativen Adjektive, also \N{’a’aw} f\"ur \D{einige},
\N{hol} f\"ur \D{wenige}, \N{pxay} f\"ur \D{viele}, \N{polpxay, holpxaype} f\"ur
\D{wieviel?}, belassen den Numerus ihre Substantive in Attributivs\"atzen
unver\"andert: \Npawl{Lu poru \uwave{’a’awa ’eylan}} \D{Er hat \uwave{einige Freunde}}.
\index{'a'aw@\textbf{’a’aw}}\index{hol@\textbf{hol}}\index{pxay@\textbf{pxay}}
\index{polpxay@\textbf{polpxay}}\index{holpxaype@\textbf{holpxaype}}
\NTeri{16/7/2010}{http://naviteri.org/2010/07/vocabulary-update/}

\subsubsection{} Im umgangssprachlichen Gebrauch kann der Numerus zusammen
mit dem Adjektiv \N{pxay}, \D{viele}, angegeben werden: \Npawl{lu awngar
\uwave{aytele apxay} a teri sa’u pivlltxe} \D{Wir haben \uwave{viele Angelegenheiten},
\"uber die wir sprechen m\"ussen.}.
\index{pxay@\textbf{pxay}!mit Substantiv im Plural}
\NTeri{16/7/2010}{http://naviteri.org/2010/07/vocabulary-update/}


\subsubsection{} Im Zusammenhang mit den Identit\"atsverben \N{lu} und \N{slu} gilt
die grundlegende Regel des Na’vi bez\"uglich der Numerusauszeichnung: "`Beim Bezug
auf dieselbe Sache geben Sie den Numerus nur einmal an."'\label{syn:noun:concord}

\begin{quotation}
\noindent\Npawl{Menga lu karyu.}, \D{Ihr beide seid Lehrer}.\\
\noindent\Npawl{Fo lu karyu.}, \D{Sie sind Lehrer}.\\
\noindent\Npawl{Menga lu oeyä ’eylan.}, \D{Ihr beide seid meine Freunde}.
\end{quotation}
\index{lu@\textbf{lu}!\"Ubereinstimmung des Numerus}
\index{slu@\textbf{slu}!\"Ubereinstimmung des Numerus}
\index{Plural!mit \textbf{lu} und \textbf{slu}}

\noindent In den ersten beiden S\"atzen wird der Numerus von \N{karyu} nicht
angegeben, da die Pronomen bereits damit ausgezeichnet sind. Gleiches gilt f\"ur
\N{’eylan} im dritten Satz. Aber siehe auch \horenref{syn:pron:q-number} f\"ur das
Fragepronomen \N{tupe}.
\NTeri{30/7/2011}{http://naviteri.org/2011/07/number-in-na\%E2\%80\%99vi/}

\subsubsection{} Allgemeine Aussagen \"uber eine Gruppe oder Kategorie werden
im Singular get\"atigt: \Npawl{Nantang\`il yom yerikit},
\D{Natterw\"olfe fressen Hexapeden}.\index{allgemeine Aussagen}
\NTeri{30/7/2011}{http://naviteri.org/2011/07/number-in-na\%E2\%80\%99vi/}

\subsection{Unbestimmtheit} Das Adjektiv \N{lahe}, \D{weitere/-r/-s, andere/-r/-s},
erf\"ahrt eine Sinnverschiebung zu \D{andere/-r/-s (im Sinne von "`unterschiedlich"')},
wenn es zusammen mit unbestimmten Substantiven, die auf \N{-o} enden, verwendet
werden, wie in \Npawl{Lu law \uwave{’uo alahe}, ma eylan} \D{\uwave{Irgendetwas
anderes} ist gewi\ss{}, meine Freunde}.
\index{Unbestimmte Substantive}\index{lahe@\textbf{lahe}!mit unbestimmten Substantiven}

\subsection{Apposition} Substantive, die in Apposition\footnote{Substantive werden
als \textit{in Apposition befindlich} beschrieben, wenn sie direkt nebeneinander
stehen, wobei eines das andere n\"aher beschreibt oder definiert. Im Gegensatz zum
Deutschen, das die beiden W\"orter einfach nebeneinander plaziert, werden die
Substantive im Englischen durch Kommata voneinander getrennt, wie in \D{I told
my best friend, Bob, that he should learn Na’vi, too}.} zu anderen Substantiven
stehen, stehen im Nominativ: \Npawl{Ayl\`i’ufa \uwave{awngey\"a ’eylan\"a a’ewan
Markus\`i}}, \D{In den Worten \uwave{unseres jungen Freundes, Markus}}. Die Konjunktion
\N{alu} kann ebenfalls hierf\"ur verwendet werden (siehe auch \horenref{syn:conj:alu}).
\footnote{Die reine Apposition ist fr\"uhes Na’vi. \N{Alu} zu verwenden d\"urfte sich
im Hinblick auf die zuk\"unftige Anwendung als sinnvoller erweisen.}\index{Apposition}

\subsubsection{Titel} Titel fungieren als Modifikatoren f\"ur Substantive und werden
nicht dekliniert, wenn sie zusammen mit Eigennamen verwendet werden. Der Dativ von
\N{Karyu Pawl}, \D{Lehrer Paul}, ist \N{Karyu Pawlur}.\index{Fall!mit Titeln}
% http://forum.learnnavi.org/language-updates/definitive-answers-on-compound-nouns/
% Followup and details: http://forum.learnnavi.org/language-updates/compound-nounsfinal-decision!/

\subsection{Adjektivzuordnung} Attributive Adjektive werden einem Substantiv unter
Zuhilfenahme des Affixes \N{-a-} zugeordnet (siehe auch \horenref{morph:adj-attr}),
\Npawl{S\`ilpey oe, layu oeru ye’r\`in \uwave{s\`iltsana fmawn}},
\D{Ich hoffe, da\ss{} ich bald \uwave{gute Nachrichten} haben werde} oder
\Npawl{Lora ayl\`i’u, lora ays\"afpìl}, \D{Sch\"one Worte sch\"one Gedanken}.
\index{Adjektiv!attributiv}\label{syn:adj:attr}

\subsubsection{} Unabh\"angig von der Reihenfolge von Substantiv und Adjektiv
werden s\"amtliche Fallendungen an das Substantiv und niemals an das Adjektiv
geh\"angt. Genauso wird eine enkliktische Adposition immer an das Substantiv
geh\"angt (siehe auch \horenref{syn:adp:position}).

\subsubsection{} Wird ein Adverb zusammen mit einem attributiven Adjektiv verwendet,
darf es nicht zwischen dem Adjektiv und dem zugeh\"origen Substantiv stehen. Also
entweder \Npawl{s\`ikenong ah\`ino n\`ihawng} oder
\Npawl{n\`ihawng h\`inoa s\`ikenong}, \D{sehr detaillierte Beispiele}, aber niemals
*\N{h\`ino n\`ihawnga s\`ikenong}.\index{Adjektiv!attributiv!mit Adverb}

\subsubsection{} Wenn zwei Adjektive ein Substantiv ver\"andern, dann neigt Frommer
dazu, diese als (Adj.) - (Subst.) - (Adj.) anzuordnen: \Npawl{N\`iawnomum tolel oel
ta ayhapx\`itu l\`i’fyaolo’\"a \uwave{pxaya s\`i\-paw\-mit atxantsan}},
\D{Wie bekannt ist, habe ich \uwave{viele ausgezeichnete Fragen} von Mitgliedern der
Sprachcommunity bekommen}.

\subsubsection{} Werden mehr als zwei Adjektive verwendet, oder soll eine andere
Reihenfolge als das oben beschriebene (Adj.) - (Subst.) - (Adj.) zum Einsatz kommen,
m\"ussen die Adjektive in einem attributiven Nebensatz mit \N{lu} plaziert werden:
\Npawl{yayo a lu lor s\`i h\`i’i}, \D{ein kleiner, h\"ubscher Vogel}.
\Ultxa{2/10/2010}{http://wiki.learnnavi.org/index.php/Canon/2010/UltxaAyharyu\%C3\%A4\%23Multiple_Attributives}

\subsubsection{} Wird ein Substantiv mit verschiedenen Adjektiven wiederholt (der
gro\ss{}e Hund, der kleine Hund, der kl\"affende Hund, usw.), so kommt das
abstrakte Substantiv\footnote{Frommer bezeichnet es als "`Blindwort"', aber es kann
ebenso als eine Art Pronomen aufgefa\ss{}t werden.} \N{pum} f\"ur die Wiederholungen
zum Einsatz: \Npawl{Lam set fwa Sawtute akawng holum, \uwave{pum as\`iltsan} ’\`i’awn},
\D{Es scheint, da\ss{} jetzt, da die b\"osen Himmelsmenschen gegangen sind, die guten
\"ubrig bleiben}.\footnote{Dieses "`Blindwort"' l\"a\ss{}t sich im Deutschen jedoch
nicht nachbilden, sondern im Gegensatz zum Englischen (welches \"uber das abstrakte
Substantiv \textit{one} verf\"ugt) bleibt das Adjektiv in dem Nebensatz alleine stehen
und wird nicht substantiviert.}\label{syn:pum:adj}
\index{pum@\textbf{pum}!mit attributiven Adjektiven}

\subsubsection{} Das Substantiv eines mit \N{si} konstruierten Verbs kann \"uber
ein attributives Adjektiv verf\"ugen: \N{wina uvan si}, \D{ein schnelles Spiel spielen}.
\index{si-Konstruktion@\textbf{si}-Konstruktion!mit attributivem Adjektiv}


\subsection{Pr\"adikation} Adjektiv- und Nominalpr\"adikate verwenden beide dieselbe
Konstruktion mit dem Verb \N{lu}, \D{sein}, wie
\Npawl{\uwave{Reltseotu atxantsan lu} nga},
\D{Du bist ein ausgezeichneter K\"unstler} oder
\Npawl{D\`isyulang lu rim}, \D{Diese Blume ist gelb}.
\index{Adjektiv!Pr\"adikation}\index{Substantiv!Pr\"adikation}\label{syn:predicates}

\subsubsection{} Andere Verben, die die pr\"adikative Satzstellung verwenden:
\N{slu} f\"ur \D{werden} und \N{’efu}, \D{f\"uhlen}, beispielsweise bei
\N{Ngenga sl\`iyu Na’viy\"a hapx\`i}, \D{Ihr werdet ein Teil des Volkes werden} oder
\N{Oe ’efu ohakx}, \D{Ich bin (f\"uhle mich) hungrig}.
\index{slu@\textbf{slu}!pr\"adikative Satzstellung}
\index{'efu@\textbf{’efu}!pr\"adikative Satzstellung}

\subsubsection{} Sollte es im Zusammenhang mit \N{slu}, \D{werden}, nicht eindeutig
sein, welches Satzglied das Subjekt und welches pr\"adikativ eingesetzt wird, kann das
Pr\"adikat mit der Adposition \N{ne-}, wie in \N{Taronyu slu \uwave{ne tsamsiyu}},
\D{Der J\"ager wird \uwave{ein Krieger}}.
\label{syn:predicate:slu-ne}\index{ne@\textbf{ne}!mit \textbf{slu}}
\index{slu@\textbf{slu}!Pr\"adikative Satzstellung mit \textbf{ne}}

\subsubsection{} \N{Sleyku}, der Kausativ von \N{slu}, \D{werden}, verwendet ebenfalls
ein adjektivisches Pr\"adikat: \Npawl{Fula tsayun oeng piv\"angkxo ye’r\`in
ulte ngari oel mokrit stayawm, \uwave{oeti nitram sleyku n\`itxan}},
\D{\uwave{Es macht mich sehr gl\"ucklich}, da\ss{} wir beide bald in der Lage sein
werden, uns zu unterhalten und da\ss{} ich Deine Stimme h\"oren werde}.
\index{sleyku@\textbf{sleyku}!pr\"adikative Satzstellung}

\subsection{Vergleiche} Adjektivische Komparative und Superlative (\D{gro\ss{},
gr\"o\ss{}er, am gr\"o\ss{}ten}) werden mit dem Partikel \N{to} versehen, das wie
eine Adposition vor dem Substantiv, mit dem verglichen wird, stehen oder an
dieses angeh\"angt werden kann (\horenref{l-and-s:stress:enclisis}).
\index{to@\textbf{to}}\index{Adjektiv!Komparativ}

\begin{quotation}
\noindent\N{Oe \uwave{to nga} lu koak}, \D{Ich bin \"alter \uwave{als Du}}.\\
\noindent\N{Oe \uwave{ngato} lu koak} \D{Ich bin \"alter \uwave{als Du}}.
\end{quotation}

\subsubsection{} Der Superlativ wird mit \N{\ACC{fra}to}, \D{als alle(s)}
konstruiert: \Npawl{F\`isyulang arim lu h\`i’i frato},
\D{Diese gelbe Blume ist die kleinste von allen}.\index{frato@\textbf{frato}}
\index{Adjektiv!Superlativ}

\subsubsection{} Gleichheitsaussagen wie "`so gro\ss{} wie ein Baum"' werden mit
der Redewendung \N{n\`iftxan} (Adjektiv) \N{na} (Substantiv oder Pronomen)
gebildet, wie in \Npawl{Oe lu n\`iftxan s\`iltsan na nga},
\D{Ich bin so gut wie Du}. Wenn der Bezugspunkt eines Vergleichs ein Pronomen oder
bestimmtes Substantiv ist, das bereits Bestandteil einer Rede ist, kann der thematische
Fall verwendet werden: \Npawl{Ngari lu oe n\`iftxan s\`iltsan}. Diese
Konstruktion l\"a\ss{}t sich auch mit Adverben verwenden.\label{syntax:adj-eql-comp}
\index{Adjektiv!Gleichheitsvergleich}\index{nìftxan@\textbf{n\`iftxan}}
\index{Fall!thematischer!Bezugspunkt des Vergleichs}
\index{na@\textbf{na}!Bezugspunkt des Vergleichs}
%% CITE: http://wiki.learnnavi.org/index.php/Canon/2010/October-December#As_ADJ.2FADV_as_N.2FPRN


\subsection{Direkte Anrede} Spricht man eine Person direkt an, so wird dem
Beziehungssubstantiv, Nominalphrase oder Namen der Vokativpartikel \N{ma}
vorangestellt: \Npawl{Oel ayngati kameie, ma oey\"a eylan},
\D{I Sehe euch, meine Freunde} oder
\Nfilm{Ma Tsu’tey, kempe si nga?}, \D{Tsu’tey, was machst du da?}
\index{Vokativ}\index{Direkte Anrede}\index{ma@\textbf{ma}}

\subsubsection{} Werden mehrere Personen angesprochen, so wird das \N{ma} nicht
wiederholt: \N{ma smukan s\`i smuke} \D{Br\"uder und Schwestern}.

\subsubsection{} An Kollektiva wird der Suffix \N{-ya} angeh\"angt, wie
\N{mawey, Na’viya, mawey} \D{Ruhig, Leute, ruhig!}
\index{-ya@\textbf{-ya}!Vokativ}


\section{Pronomen}

%\subsection{Animacy}

\subsection{Geschlecht} Die geschlechtsspezifischen Pronomen \N{poan} und \N{poe}
werden nur dann verwendet, wenn sie Mehrdeutigkeiten in einer Rede zu vermeiden
helfen. Sprecher des Deutschen oder anderer westeurop\"aischer Sprachen sollten
aufpassen, diese nicht zu oft zu benutzen.\label{syn:pron:gender}

\subsection{Numerus} Die Formen des Fragepronomens \N{tupe} zeigen ein von den
Vereinbarungen zum Numerus, die in \horenref{syn:noun:concord} diskutiert wurden,
abweichendes Verhalten.\label{syn:pron:q-number}
In diesem Fall kann das Pronomen selbst dann den Numerus anzeigen, auch wenn das
Substantiv bereits entsprechend ausgezeichnet worden ist. Beachten Sie daher die
Antworten zu diesen Fragen:

\begin{quotation}
\noindent\Npawl{Tsaysamsiyu lu \uwave{tupe}?} --- \D{Wer sind diese Krieger?}\\
\noindent\N{(Fo) lu ’eylan Tsu’tey\"a.} --- \D{Sie sind Tsu’teys Freunde.}\\

\noindent\N{Tsaysamsiyu lu \uwave{supe}?} --- \D{Wer sind diese Krieger?}\\
\noindent\N{(Fo) lu Kamun, Ralu, \`Istaw, s\`i Ateyo.} ---\\
\indent\D{Das sind Kamun, Ralu, \`Istaw und Ateyo.}
\end{quotation}

\noindent Die Pluralform fragt nach der Identit\"at einzelner Individuen, w\"ahrend
die Singularform nach einer Eigenschaft einer Gruppe fragt.
% http://naviteri.org/2011/07/number-in-na’vi/

\subsection{Fko} Das unbestimmte Pronomen \N{fko} \"ahnelt vom Prinzip her dem
deutschen \D{man}, wie in \D{Man sagt sowas nicht} oder
\Npawl{Tsat ke tsun fko yivom}, \D{Man kann das nicht essen}.
\index{fko@\textbf{fko}}

\subsubsection{} \N{Fko} wird auch dann verwendet, wenn im Deutschen das "`man"'
eingesetzt wird, um allgemeine Aussagen zu t\"atigen, wie
\Npawl{\uwave{plltxe fko} san ngaru lu mowan Txilte ulte poru nga}
\D{Man sagt, da\ss{} Du Txilte magst und umgekehrt}.\footnote{Frommers
\"Ubersetzung hierf\"ur ist \D{Ich h\"orte, da\ss{} Du Txilte magst und
umgekehrt}.}

\subsubsection{} \N{Fko} kann auch anstatt des im Deutschen vorhandenen Passivs
verwendet werden, wenn der Handelnde\footnote{Der Handelnde eines Verbs im Passiv ist
die Person oder der Gegenstand, der die Pr\"aposition "`von"' im Deutschen bekommt,
wie in \D{Ich wurde \uwave{von dem Auto} angefahren}.}, auf das sich das Verb bezieht,
lebendig ist, wie in der Redewendung \N{Oeru syaw fko W\`ily\`im},
\D{Mein Name ist Wilhelm, man nennt mich Wilhelm} oder \Npawl{Tsal\`i’uri
fko pamrel si fyape?}, \D{Wie wird das Wort geschrieben, wie schreibt man das Wort?}
\label{syn:prn:fko}
\index{Passiv!mit \textbf{fko}}\index{fko@\textbf{fko}!f\"ur den Passiv im Deutschen}

\subsection{Sno} Das Reflexivpronomen \N{sno} bezieht sich auf das Subjekt eines
Satzes. Im Genitiv wird es als \D{sein/ihr/deren/usw. eigene/-r/-s} \"ubersetzt.
Es wird verwendet, um Situationen wie in dem Satz \D{Er bereitete sein Essen zu}
aufzul\"osen. Ohne diese Verdeutlichung ist es nicht klar ersichtlich, ob sich
"`sein"' auf die Person bezieht, die das Essen zubereitet oder auf jemanden anderes.

\begin{quotation}
\noindent\N{Pol 'olem pey\"a wutsot}, \D{Er kochte sein (jemandes anderen) Essen zu.}\\
\noindent\N{Pol 'olem sney\"a wutsot}, \D{Er kochte sein eigenes Essen.}
\end{quotation}

\noindent Bisher hat Frommer lediglich Formen von \N{sno} zugestimmt, die sich
auf Bezugselemente in der dritten Person beziehen.\index{sno@\textbf{sno}}

\subsection{Lahe} Das Adjektiv \N{lahe}, \D{andere/-r/-s, weitere/-r/-s} kann, wenn
es alleine steht, auch als Pronomen verwendet werden:
\Nfilm{f\`ipoti oel tsp\`iyang, fte t\`ikenong liyevu \uwave{aylaru}}
\D{Ich werde ihn als Lektion f\"ur die anderen t\"oten} (siehe auch \horenref{morph:lahe:dat-pl}
f\"ur die Bildung).
\index{lahe@\textbf{lahe}!als Pronomen}

\subsection{Fallenlassen von Pronomen} Das Subjektpronomen (Nominativ oder Agens)
kann fallengelassen werden, wenn es sich um das gleiche Subjekt handelt wie in der
vorigen Aussage. Beachten Sie das Fehlen des Subjektpronomens im zweiten Satz:
\index{Fallenlassen von Pronomen}

\begin{quotation}
\noindent\Npawl{Fayupxarem\`i \uwave{oe} pay\"angkxo teri horen l\`i’fyay\"a
leNa’vi fpi sute a tsun srekrr tsat sivar. Ayngey\"a s\`ipawm\`iri kop
fmayi f\`itsenge tiv\`ing s\`i’eyngit.}

\medskip
\noindent\D{In diesen Mitteilungen werde \uwave{ich} f\"ur Leute, die sie bereits
anwenden k\"onnen, \"uber die Regeln der Sprache der Na’vi sprechen. \uwave{Ich}
werde auch versuchen hier Antworten auf eure Fragen zu geben.}
\end{quotation}


\section{Anwendung der F\"alle}
\label{syn:cases}
\subsection{Nominativ}\footnote{Im Englischen wird dieser Fall als der "`subjective"'
bezeichnet, aber aufgrund der \"Ahnlichkeit firmiert er hier als Nominativ.} Der
nichtmarkierte Nominativ bildet das Subjekt intransitiver Verben, das Pr\"adikatsnomen
in pr\"adikativen Konstruktionen (\horenref{syn:predicates}) und zusammen mit
Adpositionen.\index{Fall!Nominativ}\index{Nominativ}

\subsubsection{} Folgt die Zielangabe direkt auf ein Verb der Bewegung, kann die
Adposition \N{ne-} fallengelassen werden, so da\ss{} das unver\"anderte Substantiv
\"ubrig bleibt.\Npawl{Za’u \uwave{f\`itseng}, ma ’itetsy\`ip},
\D{Komm \uwave{her}, T\"ochterlein}.\label{syn:subjective:ne}
\index{ne@\textbf{ne}!fallengelassen mit Verben der Bewegung}

\subsubsection{} Der Nominativ wird auch dann verwendet, wenn ein Substantiv oder eine
Nominalphrase als einziges verwendet wird, beispielsweise in einem Ausruf:
\Npawl{Lora ayl\`i’u, lora ays\"afp\`il},
\D{Sch\"one Worte, sch\"one Gedanken}
\Npawl{ayl\`i’u apawnlltxe n\`iltsan}, \D{Gut gesprochene Worte!}
\index{Fall!Nominativ!Ausrufe}

\subsubsection{} Ein Zeitwort, das mit dem Unbestimmtheits-\N{o} zusammen verwendet
wird, steht im Nominativ, um eine Zeitdauer anzugeben.
\Nfilm{\uwave{Z\`is\`ito amrr} ftolia ohe}, \D{Ich habe \uwave{f\"unf Jahre lang} studiert},
\Npawl{Herw\`i zereiup \uwave{f\`itrro n\`iwotx}!} \D{Es schneit schon \uwave{den Ganzen Tag}!}
\index{-o@\textbf{-o}!in Zeitausdr\"ucken}
%% http://wiki.learnnavi.org/index.php/Canon/2010/October-December#Duration_and_Loan_Word.2C_.22Jesus.22

\subsection{Agens} Der Agens bildet das Subjekt transitiver Verben.
\N{\uwave{Oel} ngati kameie}, \D{Ich Sehe Dich}.
\index{Fall!Agens}\index{Subjekt}

\subsection{Patiens} Der Patiens bildet das direkte Objekt transitiver Verben.
\Npawl{\uwave{T\`i’eyngit} oel tolel a krr}, \D{Wenn ich eine Antwort erhalte}.
\index{Fall!Patiens}\index{Direktes Objekt}

\subsection{Dativ} Der Dativ bildet das indirekte Objekt sowohl von intransitiven
als auch einigen transitiven Verben: \Npawl{S\`iltsana fmawn a tsun oe \uwave{ayngaru}
tiv\`ing}, \D{Gute Nachrichten, die ich \uwave{euch} geben kann}.
\index{Fall!Dativ}\index{Indirektes Objekt}

\subsubsection{} Das Objekt eines mit \N{si} konstruierten Verbs steht grunds\"atzlich
im Dativ: \N{Oe irayo si ngaru}, \D{Ich danke Dir.} \label{syn:case:dative-si}

\subsubsection{} Der "`Verursachte"', der zum Kausativ eines transitiven Verbs geh\"ort,
kann im Dativ stehen: \N{Oel \uwave{ngaru} tseyk\-\`iy\-e’a tsat},
\D{I werde daf\"ur sorgen, da\ss{} \uwave{Du} es siehst} (Siehe auch \horenref{syn:trans-causative}).
\index{Fall!Dativ!mit Kausativ}

\subsubsection{} Das Verb \N{lu} ergibt zusammen mit dem Dativ eine Redewendung, die
Besitz ausdr\"uckt, wo das Deutsche das Verb "`haben"' hat: \N{Lu oeru ikran},
\D{Ich habe einen Ikran}. In dieser Konstruktion steht das Verb im Regelfalle an
erster Stelle im Satz.\index{Fall!Dativ!mit \textbf{lu}}
\index{lu@\textbf{lu}!mit Dativ}
\LNWiki{28/1/2010}{http://wiki.learnnavi.org/index.php/Canon\%23Dative_.2B_copula_possessive}

\subsubsection{} Der Dativ, der Interesse ausdr\"uckt, schr\"ankt den Rahmen
eines Adjektivs auf die Beurteilung \QUAESTIO{oder den Vorteil} einer bestimmten
Person: \N{F\`i’u oeru prrte’ lu}, \D{Dies ist mir angenehm} oder
\N{t\`ip\"angkxo ayoengey\"a mowan lu oeru n\`ingay},
\D{Unsere Unterhaltung ist (mir) sehr angenehm}.

\subsubsection{} F\"ur Verben des Sprechens (inklusive \N{pawm}, \D{fragen}) steht
die angesprochene Person im Dativ: \N{Oel poru polawm f\`i’ut},
\D{Ich habe ihn dies gefragt}.
\index{Fall!Dativ!mit Verben des Sprechens}

\subsection{Genitiv} Der Genitiv zeigt Besitz an, wie in \N{oey\"a ’eylan},
\D{mein Freund}.\index{Fall!Genitiv}
Aber siehe unten f\"ur unver\"au\ss{}erlichen Besitz (\horenref{syn:topical:poss}).
\index{Possessiv}

\subsubsection{} Der Genitiv kann pr\"adikativ verwendet werden, so wie in
\N{F\`itseng lu awngey\"a}, \D{Dieser Ort ist unser}. Jedoch wird das abstrakte
Substantiv \N{pum}, \D{Besitz, besessenes Objekt}, h\"aufiger verwendet:
\Npawl{Kelku ngey\"a lu tsawl; \uwave{pum oey\"a} lu h\`i’i},
\D{Dein Haus ist gro\ss{}, \uwave{meins} ist klein}.\label{syn:pum:genitive}
\index{pum@\textbf{pum}!mit Genitiv}

\subsubsection{} Der partitive Genitiv gibt an, zu welchem gr\"o\ss{}eren Ganzen etwas
geh\"ort: \Nfilm{Na’viy\"a luyu hapx\`i}, \D{Du bist ein Teil des Volkes}. Er wird ebenfalls
zusammen mit Br\"uchen verwendet.
\Npawl{Tsu’tey\`il tolìng oer mawlit \uwave{smar\"a}} \D{Tsu’tey hat mir die H\"alfte
\uwave{der Beute} gegeben}. 
\index{Fall!Genitiv!partitiv}\label{syn:partitive-gen}

\subsubsection{} Der Genitiv wird gelegentlich von dem Substantiv, zu dem er geh\"ort,
abgetrennt: \Nfilm{Na’viy\"a luyu hapx\`i} 
\D{Du bist ein Teil des Volkes}.\index{Fall!Genitiv!Verschiebung}

\subsubsection{} Der Genitiv wird auch als Objekt f\"ur aus Verben konstruierten
Substantiven verwendet, wie in \Npawl{t\`iftia kifkey\"a} \D{Studium der nat\"urlichen Welt}.

\subsection{Thematisch} Der thematische Fall bezeichnet das Thema in einer
Themen-Kommentarkonstruktion. Siehe auch \textit{Thema-Kommentar},\horenref{pragma:topic-comment},
f\"ur eine umfassendere Diskussion zu dieser Verwendungsm\"oglichkeit.
Der thematische Fall verf\"ugt \"uber ein paar weitere feste Verwendungsm\"oglichkeiten.
\index{Fall!thematisch}

\subsubsection{} In Prosa wird ein themenbezogenes Substantiv m\"oglichst fr\"uhzeitig im
Satz auftauchen: An erster Stelle in einem Hauptsatz und direkt hinter der Konjunktion in
einem Nebensatz.\label{syn:topical:word-order}
\LNWiki{8/10/2011}{http://wiki.learnnavi.org/index.php/Canon/2011/April-December\%23Topical_Position}

\subsubsection{} Der thematische Fall wird oftmals zusammen mit einem mit \N{si} konstruierten
Verb verwendet --- \N{irayo si}, \D{danken} --- um anzugeben, wof\"ur sich bedankt wird.
\Npawl{\uwave{T\`imweypeyri ayngey\"a} seiyi irayo n\`i\-ngay}, \D{Ich danke Dir wirklich
\uwave{f\"ur Deine Geduld}}.

\subsubsection{} Dieser Fall kann verwendet werden, um unver\"au\ss{}erlichen Besitz
anzuzeigen\footnote{Unver\"au\ss{}erlicher Besitz ist Besitz derjenigen Dinge, die spezifisch
f\"ur eine bestimmte Person sind und die theoretisch nicht weitergegeben oder weggenommen
werden k\"onnen. In den meisten Sprachen, die dar\"uber verf\"ugen, sind die Blutsverwandten
die wahrscheinlichsten Personen, die \"uber eine spezielle Grammatik in Bezug auf
unver\"au\ss{}erlichen Besitz verf\"ugen. Das Na’vi schlie\ss{}t K\"orperteilemit ein, was in
verschiedenen menschlichen Sprachen ebenfalls recht h\"aufig ist.}: \Npawl{\uwave{Oeri}
n\`i’i’a \uwave{tsyokx} zoslolu}, \D{\uwave{Meine Hand} ist endlich geheilt} oder
\Npawl{Oeri t\`ingay\`il txe’lanit tivakuk}, \D{La\ss{} die Wahrheit mein Herz treffen},
aber auch \Npawl{\uwave{Ngari tswintsy\`ip} sevin n\`itxan lu nang!} \D{Welch h\"ubschen,
kleinen Tswin Du hast!}
Beachten Sie in den ersten beiden Beispielen, da\ss{} das besessene Substantiv nicht direkt
an den Themenbezug angrenzen mu\ss{}.\index{Besitz!unver\"au\ss{}erlich}\label{syn:topical:poss}
\index{Fall!Thematisch!unver\"au\ss{}erlicher Besitz}
% oeri ta peyä fahew akewong ontu teya längu.

\subsubsection{} Der thematische Fall kann auch verwendet werden, um den Bezugspunkt
eines Vergleichs auf Gleichheit zu bezeichnen (siehe auch \horenref{syntax:adj-eql-comp}).


\section{Adpositionen}\index{Adpositionen}
\noindent Die Adpositonen des Na’vi beeinflussen Substantive, Pronomen und Adverben des
Ortes und der Zeit.

\subsection{Position} Eine Adposition kann an zwei Stellen auftreten: Vor der Nominalphrase,
den sie ver\"andern soll, als getrenntes Wort (\Npawl{\uwave{Ta} pey\"a fahew akewong},
\D{\uwave{Von} seinem fremdarigen Geruch} oder \Nfilm{Ngari \uwave{hu Eywa}
salew tirea}, \D{Dein Geist geht \uwave{mit Eywa}}) oder aber enkliktisch, wobei sie an
das zu ver\"andernde Wort angeh\"angt wird: \Npawl{F\`itrr\uwave{m\`i} letsranten},
\D{An diesem wichtigen Tag} oder \Npawl{Ayl\`i’u\uwave{fa} awngey\"a ’eylan\"a a’ewan},
\D{Mit den Worten unserer jungen Freunde}.\index{Adpositionen!Position}\label{syn:adp:position}

\subsection{Lenition}\index{Plural!kurz} Sieben der Adpositionen plus zwei zusammengesetzte
Adpositionen mit \N{sre} verursachen Lenition des folgenden Wortes. Dies wird durch ein
Pluszeichen in Klammern, \N{(+)}, in der untenstehenden Liste angezeigt.

\subsubsection{} Angeh\"angte Adpositionen verursachen keine Lenition.

\subsubsection{} Da Lenition f\"ur sich auch f\"ur den kurzen Plural
(\horenref{morph:short-plural}) verwendet wird, besteht die Wahrscheinlichkeit, da\ss{}
bez\"uglich des Numerus eine Ungewi\ss{}heit entsteht, je nach Kontext einer Unterhaltung.
Um den Numerus klarzustellen, verwenden Sie den Pluralpr\"afix; die lenisierte Form ohne
\N{ay+} sollte dann als Einzahl interpretiert werden.
\index{Plural!kurz!mit lenisierenden Adpositionen}\label{syn:adp:short-plural}
\NTeri{1/7/2010}{http://naviteri.org/2010/07/thoughts-on-ambiguity/}

\subsubsection{} Das Wort, das direkt auf eine nichtenkliktische Adposition folgt,
wird leniert. Das mu\ss{} nicht zwingend das Substantiv sein: \N{M\`i hivea trr},
\D{Am siebten Tag} (und nicht *\N{m\`i kivea srr}).
\LNWiki{24/8/2010}{http://wiki.learnnavi.org/index.php/Canon/2010/July-September\%23Fmawno}

\subsection{\"Ao}, \D{unter}. \N{\"Ao Utral Aymokriy\"a}, \D{unter dem Baum der Stimmen}.
\index{aäo@\textbf{\"ao}}\label{syn:adp:äo}

\subsection{Eo} \D{vor} (Ort). Kann im \"ubertragenen Sinn verwendet werden.
\Nfilm{Eo ayoeng lu txana t\`ikawng}, \D{Ein gro\ss{}es \"Ubel ist \"uber uns gekommen} oder
\Npawl{Tokx eo Tokx}, \D{von Angesicht zu ANgesicht, pers\"onlich}.
\index{eo@\textbf{eo}}\label{syn:adp:eo}

\subsection{Fa} \D{mit, durch} Bitte nicht mit \N{hu} verwechseln (\horenref{syn:adp:hu}).
\index{fa@\textbf{fa}}\label{syn:adp:fa}

\subsubsection{} \N{Fa} kann verwendet werden, um zu zitierende Worte anzuk\"undigen:
\Npawl{\uwave{Ayl\`i'ufa} awngey\"a ’eylan\"a a’ewan} \D{\uwave{Mit den Worten} unserer
jungen Freunde}.

\subsubsection{} \N{Fa} ist eine weitere M\"oglichkeit, den "`Verursachten"' anzugeben,
wenn ein transitives Verb im Kausativ steht (siehe auch \horenref{syn:trans-causative}).

\subsection{Few} \D{(hin)\"uber, (zur) andere(n) Seite}. Bitte nicht mit \N{ka}
verwechseln (\horenref{syn:adp:ka}). \Npawl{Po sp\"a few payfya fte smarit sivutx}
\D{Er sprang \"uber den Flu\ss{}, um seine Jagdbeute zu verfolgen}.
\index{few@\textbf{few}}\label{syn:adp:few}

\subsection{Fkip} \D{oben zwischen}
\index{fkip@\textbf{fkip}}\label{syn:adp:fkip}

\subsection{Fpi (+)} \D{zugunsten von}. Bezieht sich auf Personen.
\Npawl{Fayupxare layu aysng\"a\-’i\-yufpi}, \D{Diese Nachrichten sind f\"ur Anf\"anger}
oder Gegenst\"ande: \Npawl{’Uo a fpi rey’eng Eywa\-’e\-veng\-m\`i ’Rrtam\`i tsranten
n\`itxan awngaru n\`iwotx}, \D{Etwas, das uns alle etwas angeht um der Balance des Lebens
willen sowohl auf Pandora als auch auf der Erde}.
\index{fpi@\textbf{fpi}}\label{syn:adp:fpi}

\subsection{Ftu} \D{von, aus} (Richtung).
Diese wird meist mit Verben willentlicher Bewegung verwendet, wie \N{k\"a}, \N{rikx}, usw.
\index{ftu@\textbf{ftu}}\label{syn:adp:ftu}  Siehe \N{ta} weiter unten.

\subsection{Hu} \D{mit}. Nur bezogen auf Begleiter --- Bitte nicht mit \N{fa} verwechseln
(\horenref{syn:adp:fa}). \N{Tsun oe ngahu piv\"angkxo a f\`i’u oeru prrte’ lu},
\D{Es ist mir ein Vergn\"ugen, mich mit Dir unterhalten zu k\"onnen.}
\index{hu@\textbf{hu}}\label{syn:adp:hu}

\subsection{Io} \D{\"uber}. \Npawl{Kllkxayem f\`it\`ikangkem oey\"a rofa — ke io — pum fey\"a},
\D{Diese meine Arbeit steht neben — und nicht \"uber — deren}.
\index{io@\textbf{io}}\label{syn:adp:io}

\subsection{\`Il\"a (+)} \D{folgend, \"uber (Wegepunkt), entlang}. \Npawl{Rerol tengkrr ker\"a
/ \`Il\"a fya’o avol / Ne kxamtseng}, \D{Wir singen, w\"ahrend wir entlang der acht Pfade zum
Zentrum gehen} oder \Npawl{Ayfo solop ìlä hilvan fa uran}, \D{Sie fuhren in einem Boot den
Flu\ss{} entlang}.
\index{ilä@\textbf{\`il\"a}}\label{syn:adp:ìlä}

\subsection{Ka} \D{\"uber, zur\"ucklegend}. Bitte nicht mit \N{few} verwechseln
(\horenref{syn:adp:few}).
\index{ka@\textbf{ka}}\label{syn:adp:ka}

\subsection{Kam} \D{vor (Vergangenheit)}. \Npawl{Tskot sngol\"a’i po sivar ’a’awa trrkam}
(oder \N{kam trr a’a’aw}), \D{Er fing vor einigen Tagen an, seinen Bogen zu benutzen}.
\index{kam@\textbf{kam}}\label{syn:adp:kam}
\NTeri{24/9/2011}{http://naviteri.org/2011/09/miscellaneous-vocabulary/}

\subsection{Kay} \D{in (Zukunft)}. \Npawl{Zaya’u Sawtute fte awngati skiva’a kay z\`is\`it apxey}
(oder \N{pxeya z\`is\`itkay}), \D{Die Himmelsmenschen werden in drei Jahren kommen, um uns zu
vernichten}!
\index{kay@\textbf{kay}}\label{syn:adp:kay}
\NTeri{24/9/2011}{http://naviteri.org/2011/09/miscellaneous-vocabulary/}

\subsection{Kip} \D{zwsichen, unter}. \Nfilm{Tiv\`iran po ayoekip},
\D{La\ss{} sie unter uns wandeln}.
\index{kip@\textbf{kip}}\label{syn:adp:kip}

\subsection{Kxaml\"a} \D{durch (die Mitte von)}.
\index{kxamlä@\textbf{kxaml\"a}}\label{syn:adp:kxamlä}

\subsection{Lisre (+)} Siehe auch \N{li}, \horenref{syn:li:sre}.

\subsection{Lok} \D{nahe bei}.
\index{lok@\textbf{lok}}\label{syn:adp:lok}

\subsection{Luke} \D{ohne}. \Npawl{Luke pay, ke tsun ayoe t\`ireyti fmival},
\D{Ohne Wasser k\"onnen wir nicht \"uberleben}. Dies kann auch mit substantivierten W\"ortern
verwendet werden (siehe auch \horenref{syn:rel:nom-adp}).
\index{luke@\textbf{luke}}\label{syn:adp:luke}

\subsection{Maw, Pximaw} \D{nach (time)}. \Npawl{Maw h\`ikrr ayoe t\`iy\"atxaw},
\D{Wir werden in K\"urze zur\"uck sein}.
\index{maw@\textbf{maw}}\label{syn:adp:maw}
\index{pximaw@\textbf{pximaw}}\label{syn:adp:pximaw}

\subsection{M\`i (+)} \D{in, an, auf}. Zeigt an, da\ss{} jemand oder etwas an einem Ort ist.
Bewegung darin oder hinein ist \N{nemfa}.
\index{miì@\textbf{m\`i}}\label{syn:adp:mì}

\subsubsection{} \N{M\`i} beschreibt Stellen im oder am K\"orper: \N{Ayl\`i’u na ayskxe
\uwave{m\`i te’lan}}, \D{Die Worte (sind) wie Steine \uwave{in meinem Herzen}} (aus dem
Drehbuch). \N{\uwave{M\`i tal} ngey\"a prrnen\"a a sanh\`i lor n\`itxan lu nang},
\D{Welch h\"ubsche Sterne [Leuchtpunkte, Anm. d. \"Ubers.] Dein Baby \uwave{auf seinem R\"ucken}
hat}. Es beschreibt auch einen Ort in (oder "`auf"') einem Planeten: \Npawl{L\`i’fyari leNa’vi
\uwave{’Rrtam\`i}, vay set ’almong a fra’u zera’u ta ngrrpongu}, \D{Alles, was bisher mit dem
Na’vi auf der Erde passiert ist, kam von einer Basisbewegung}.

\subsubsection{} Es kann auch verwendet werden, um einen zeitlichen Bezug auszudr\"ucken:
\Npawl{f\`itrr\uwave{m\`i} letsranten} \D{an diesem wichtigen Tag}.

\subsubsection{} \QUAESTIO{Wie erkl\"art man das: \Npawl{Law lu oeru fwa nga \uwave{m\`i
reltseo} nolume n\`itxan!} Begrenzungen im Anwendungsbereich, wie bei \N{m\`i s\`irey}?}

\subsubsection{} Andere Redewendungen mit \N{m\`i}: \N{t`i’efum\`i oey\"a}, \D{meiner Meinung
nach}.

\subsubsection{} Obwohl sich die Schreibweise nicht \"andert, verschwindet der Vokal \N{\`i}
beim Sprechen, wenn der Pluralpr\"afix \N{ay+} folgt. \N{m\`i ayhilvan} wird wie
*\N{mayhilvan} ausgesprochen (\horenref{l-and-s:elision-i}).

\subsection{M\`ikam} \D{zwischen}
\index{miìkam@\textbf{m\`ikam}}\label{syn:adp:mìkam}

\subsection{Mungwrr} \D{au\ss{}er}
\index{mungwrr@\textbf{mungwrr}}\label{syn:adp:mungwrr}

\subsection{Na} \D{wie, als, so wie}. \N{Ayl\`i’u na ayskxe m\`i te’lan},
\D{Diese Worte sind wie Steine in (meinem) Herzen}.
\index{na@\textbf{na}}\label{syn:adp:na}

\subsubsection{} \N{Na} wird benutzt, um Farbschattierungen anzugeben: \Npawl{F\`isyulang lu
\uwave{ean na ta’leng}} oder \Npawl{F\`isyulang lu \uwave{ta’lengna ean}},
\D{Diese Blume ist hautblau}. Siehe auch \horenref{syn:attr:na} f\"ur Farben in
Attributivs\"atzen mit \N{na}.

\subsubsection{} \N{Na} wird in Vergleichen auf Gleichheit verwendet, um den Bezugspunkt des
Vergleichs anzuzeigen (siehe auch \horenref{syntax:adj-eql-comp}).

\subsection{Ne} \D{zu, nach, in Richtung}. Dies gibt das Ziel von Verben der Bewegung an.
\Npawl{Ter\`iran ayoe \uwave{ayngane}}, \D{Wir gehen in Deine Richtung}. Manchmal kann \N{ne}
ausgelassen werden (siehe auch \horenref{syn:subjective:ne}).

\subsubsection{} Redewendungen mit \N{ne}: \Npawl{Ke \uwave{zasyup} l\`i’Ona \uwave{ne} kxutu
a m\`ifa fu a wrrpa}, \D{Die L\`i’Ona werden weder dem \"au\ss{}eren noch dem inneren Feind
anheimfallen} oder \Npawl{zola’u n\`iprrte’ ne p\`ilok Na’viteri}, \D{Willkommen auf dem
Na’viteri-Blog}.

\subsubsection{} \N{Ne} kann verwendet werden, um das Pr\"adikat des Verbs \N{slu}, \D{werden},
eindeutig zu machen (\horenref{syn:predicate:slu-ne}).
\index{ne@\textbf{ne}}\label{syn:adp:ne}

\subsection{Nemfa} \D{in ... hinein}.  Siehe auch \N{m\`i} (\horenref{syn:adp:mì}).
\index{nemfa@\textbf{nemfa}}\label{syn:adp:nemfa}

\subsection{Nu\"a (+)} \D{jenseits (weit entfernt)}. Beachten Sie bitte den Kontrast zwischen
\N{few}: \Npawl{Fo kelku si few ’ora}, \D{Sie leben auf der anderen Seite des Sees} und
\Npawl{Fo kelku si nu\"a ora}, \D{Sie leben jenseits des Sees} (d. h. weit entfernt und
au\ss{}erhalb der Sicht).\index{nuaä@\textbf{nu\"a}}\label{syn:adp:nuä}
\NTeri{15/8/2011}{http://naviteri.org/2011/08/new-vocabulary-clothing/comment-page-1/\%23comment-986}

\subsection{Pxaw} \D{um ... herum}. \N{Po pxaw txep srew}, \D{Er tanzte ums Feuer herum}.
\index{pxaw@\textbf{pxaw}}\label{syn:adp:pxaw}

\subsection{Pxel} \D{wie}. \Npawl{Fwa sute pxel nga tsun oey\"a h\`i’ia t\`ingopit sivar fte
pivlltxe n\`ilor f\`itxan oeru teya si}, \D{Da\ss{} Leute wie Du in der Lage sind, meine
kleine Sch\"opfung einzusetzen, um so wundervoll zu sprechen, erf\"ullt mich mit Freude}.
\index{pxel@\textbf{pxel}}\label{syn:adp:pxel}

\subsection{Ro (+)} \D{bei}.
\index{ro@\textbf{ro}}\label{syn:adp:ro}

\subsection{Rofa} \D{neben, nebenher}.  \Npawl{Kllkxayem f\`it\`ikangkem oey\"a rofa — ke
io — pum fey\"a}, \D{Diese meine Arbeit steht neben — und nicht \"uber — deren}.
\index{rofa@\textbf{rofa}}\label{syn:adp:rofa}

\subsection{S\`in} \D{auf, hinauf}. \Npawl{Aywayl y\`im kifkey\"a / ’\`Iheyut avomrr
/ S\`in tireafya’o avol}, \D{Die Lieder flechten die dreizehn Spiralen der Welt auf die
acht Pfade des Geistes}.
\index{siìn@\textbf{s\`in}}\label{syn:adp:sìn}

\subsection{Sko (+)} \D{als, in der Eigenschaft als, in der Funktion als}
\Npawl{Sko Sahìk ke tsun oe mìftxele tsngivawvìk} \D{Als Tsahìk kann ich wegen dieser
Angelegenheit nicht weinen}.
\index{sko@\textbf{sko}}\label{syn:adp:sko}
\NTeri{31/3/2012}{http://naviteri.org/2012/03/spring-vocabulary-part-2/}

\subsection{Sre (+), Pxisre (+)}  \D{vor (Zeit)}.
\index{sre@\textbf{sre}}\label{syn:adp:sre}
\index{pxisre@\textbf{pxisre}}\label{syn:adp:pxisre}

\subsection{Ta} \D{von}. \Npawl{Oeri ta pey\"a fahew akewong ontu teya l\"angu},
\D{Meine Nase ist voll von seinem fremdartigen Geruch}.
\index{ta@\textbf{ta}}\label{syn:adp:ta}

\subsubsection{} \N{Ta} gibt das Ursprungsland an: \Npawl{Markus\`i ta Ngalwey},
\D{Markus aus Galway}.

\subsubsection{} Bezogen auf die Zeit bedeutet \N{ta} \D{seit}: \Npawl{Trr’ongta
txon’ongvay po tolìran}, \D{Er lief von Sonnenauf- bis -untergang}
(siehe auch \N{takrra}, \horenref{syn:attr:takrra}).

\subsubsection{} Frommer schreibt oftmals \N{ta Pawl}, \D{von Paul}, am Ende seiner
E-Mails und Blogeintr\"age.

\subsubsection{} Zusammen mit transitiven Verben ist die Wahrscheinlichkeit h\"oher
als mit \N{ftu}, da\ss{} \N{ta} eine Bewegung ausdr\"uckt, wie in \Npawl{pot ’aku
f\`itsengta}, \D{Schafft ihn hinfort}!
\NTeri{15/8/2011}{http://naviteri.org/2011/08/new-vocabulary-clothing/comment-page-1/\%23comment-994}


\subsection{Takip} \D{von zwischen, von inmitten}.
\index{takip@\textbf{takip}}\label{syn:adp:takip}

\subsection{Tafkip} \D{von oben zwischen}.
\index{tafkip@\textbf{tafkip}}\label{syn:adp:tafkip}

\subsection{Teri} \D{\"uber, betreffend}. \Npawl{Fayupxarem\`i oe pay\"angkxo teri horen
l\`i'fyay\"a leNa’vi}, \D{In diesen Nachrichten werde ich \"uber die Regeln des Na'vi
sprechen}.
\index{teri@\textbf{teri}}\label{syn:adp:teri}

\subsection{Uo} \D{hinter}
\index{uo@\textbf{uo}}\label{syn:adp:uo}

\subsection{Vay} \D{bis, bis zu}. Diese kann sowohl f\"ur Raum als auch f\"ur Zeit
eingesetzt werden: \Npawl{Tsakrrvay, ayngey\"a t\`imweypeyri irayo seiyi oe},
\D{Bis dahin danke ich euch f\"ur eure Geduld}.
\QUAESTIO{Es gibt eine Passage mit einem lokalen Gebrauch im Videospiel.}
\index{vay@\textbf{vay}}\label{syn:adp:vay}

\subsubsection{} Der Ausdruck \N{vay set ke} bedeutet \D{noch nicht}.
\index{vay@\textbf{vay}!\textbf{vay set ke}}

\subsection{W\"a (+)} \D{gegen} (wie in "`k\"ampfen gegen"').
\Npawl{Pey\"a tsat\`ipe’un a sweylu txo wivem ayoeng Omatikayaw\"a lu fe’}
\D{Die Entscheidung, gegen die Omatikaya zu k\"ampfen, war schlecht}.
\index{waä@\textbf{w\"a}}\label{syn:adp:wä}
%http://naviteri.org/2011/07/txantsana-ultxa-mi-siatll-great-meeting-in-seattle/


\section{Adverben}
\index{Adverben}

\subsection{Ma\ss{} und Menge} Adverben der Ma\ss{}e und der Menge folgen sehr oft dem
Element, das sie ver\"andern: \Npawl{’Rrtam\`i tsranten \uwave{n\`itxan} awngaru
\uwave{n\`iwotx}}, \D{Das bedeutet uns \uwave{allen} auf der Erde \uwave{sehr viel}}.

\subsubsection{} Ein sehr g\"angiges Muster im Zusammenhang mit pr\"adikativen Adjektiven
ist (Adj.) \N{lu} (Adv.): \Npawl{Ngey\"a l\`i’fya leNa’vi txantsan \uwave{lu n\`ingay}},
\D{Dein Na’vi ist wirklich ausgezeichnet}.

\subsection{mit Gerundien} Das Gerundium beh\"alt gen\"ugend seines Ursprungs als
Verb, da\ss{} es ebenfalls mit einem Adverb versehen werden kann\footnote{Dies funktioniert
im Englischen hervorragend, da es selbst \"uber ein Gerundium verf\"ugt. Ins Deutsche l\"a\ss{}t
es sich nicht so leicht \"ubertragen und wird oftmals als Substantivierung dargestellt.}:
\Npawl{Koren a’awve \uwave{t\`irusey\"a ’awsiteng}}, \D{Die erste Regel f\"urs Zusammenleben}.
\index{Gerundium!mit Adverb}\label{syn:adverbs:gerund}

\subsection{Keng} Das Adverb \N{keng}, \D{selbst, sogar, gar}, um unerwartete Informationen
zu untermauern: \N{Yom teylut \uwave{keng oel}}, \D{\uwave{Sogar ich} esse Teylu}.
\index{keng@\textbf{keng}}
\LNWiki{31/12/2010}{http://wiki.learnnavi.org/index.php/Canon/2010/October-December\%23Keng}

\subsection{Li} Die Hauptbedeutung von \N{li} ist \D{schon, bereits}:
\Npawl{T\`ikangkem li hasey lu}, \D{Die Arbeit ist bereits fertig}.
\index{li@\textbf{li}} 
\NTeri{20/2/2011}{http://naviteri.org/2011/02/new-vocabulary-part-2/}

\subsubsection{"`Bisher nicht"', "`Noch nicht"'} Die Negation \N{ke li} bedeutet "`bisher
nicht"' und benutzt die pleonastische Negation (\horenref{syn:neg:pleon}): \N{Fo ke li ke
pol\"ahem}, \D{Sie sind noch nicht angekommen}.
\index{bisher nicht}\index{noch nicht}\index{ke@\textbf{ke}!\textbf{ke li}}
\NTeri{4/9/2011}{http://naviteri.org/2011/09/\%E2\%80\%9Cby-the-way-what-are-you-reading\%E2\%80\%9D/comment-page-1/\%23comment-1092}



\subsubsection{Befehle} Zusammen mit dem Imperativ bedeutet \N{li} h\"ochste Dringlichkeit:
\Npawl{Ngal mi f\`itsengit terok srak? Li k\"a!}, \D{Du bist immer noch hier? Los jetzt!}.
Zusammen mit it \N{ko} (\horenref{syn:particle:ko}, Betonung liegt auf \N{li}) bedeutet
es "`Komm in die G\"ange"' oder "`Beeilung!"'

\subsubsection{Z\"ogern} In einer Antwort transportiert es ein etwas z\"ogerliches "`Ja"',
etwa wie das deutsche "`sozusagen"'.

\begin{quotation}
\noindent A: \Npawl{Nga mllte srak?}, \D{Stimmst Du zu?}\\
\noindent B: \N{Li, sl\"a\dots}, \D{Nun, ja. Ich denke schon, aber\dots}.
\end{quotation}

\noindent Die Negation dessen, \N{ke li}, bedeutet in etwa "`nicht wirklich"'.

\subsubsection{Mit "`sre"'} Zusammen mit der Adposition \N{sre} bekommt es die Bedeutung
"`bis sp\"atestens"'. \Npawl{Kem si li trraysre}, \D{Erledige das bis sp\"atestens morgen}.
\label{syn:li:sre} Steht \N{sre} vor dem Substantiv, so verschmilzt es mit \N{li} zu
\N{\ACC{li}sre}, was, genauso wie \N{sre} Lenition hervorruft: \Npawl{kem si lisre srray},
\D{Erledige das bis sp\"atestens morgen}.
\index{lisre@\textbf{lisre}}


\subsection{N\`iwotx} Das Adverb \N{n\`iwotx}, \D{in toto, im Ganzen, ganz, allesamt} wird oft
zusammen mit Substantiven und Pronomen im Plural eingesetzt, um einen Kollektivsinn zu
erzeugen: \Npawl{Ayeylanur oey\"a s\`i eylanur l\`i’fyay\"a leNa’vi \uwave{n\`iwotx}},
\D{An \uwave{alle} meine Freunde und Freunde der Sprache der Na’vi} oder
\Npawl{T\`ifyaw\`intxuri oey\"a perey aynga \uwave{nìwotx}},
\D{Ihr wartet \uwave{allesamt} auf meine Anleitung}.
\index{nìwotx@\textbf{n\`iwotx}}

\subsubsection{"`Beide"'} Zusammen mit dem dualen Numerus bedeutet \N{n\`iwotx} \D{beide}:
\N{Mefo n\`iwotx yolom}, \D{Sie aßen beide}.
\index{nìwotx@\textbf{n\`iwotx}!mit Dual}\index{Beide}
\NTeri{15/8/2011}{http://naviteri.org/2011/08/new-vocabulary-clothing/}

\subsection{N\`ifya’o} Eine Nominalphrase, der mit \N{fya’o} konstruiert wird, kann
grunds\"atzlich daf\"ur eingesetzt werden, um Adverben der Art und Weise zu bilden. In diesem
Fall wird die komplette Nominalkonstruktion zu einer Adverbialkonstruktion und nicht blo\ss{}
das Wort, an dem das \N{n\`i-} h\"angt: \N{n\`i-[fya’o letrrtrr]}, \D{auf regul\"are Art und
Weise} oder \Npawl{Poe poltxe n\`ifya’o alaw}, \D{Sie sprach deutlich}.
\index{nìfya'o@\textbf{n\`ifya’o}}\label{syn:nifyao}

\subsubsection{} \N{N\`ifya’o} kann auch zusammen mit Attributivs\"atzen verwendet werden:
\Npawl{N\`ifya’o a hek}, \D{In gewisser Hinsicht ist das eigenartig}.

\subsubsection{} \QUAESTIO{Anmerkung betreffs Satzadverben gegen\"uber \N{n\`ifya’o}-Formen?}

\subsection{"`Kop"' und "`n\`iteng"'} Sowohl \N{kop} als auch \N{n\`iteng} entsprechen dem
deutschen Adverb \D{auch, ebenfalls}. Allerdings hat \N{kop} die Nebenbedeutung von
\D{dar\"uber hinaus, gleichwohl}, w\"ahrend \N{n\`iteng} eher dem \D{gleichartig,
zus\"atzlich} entspricht. Vergleichen Sie \N{Oel poleng kop poru tsa’ut}, \D{Dar\"uber hinaus
habe ich ihm das gesagt} mit \N{Oel poleng n\`iteng poru tsa’ut}, \D{Das habe ich ihm
ebenfalls gesagt}.
\index{kop@\textbf{kop}}
\index{niìteng@\textbf{n\`iteng}}

\subsubsection{} Beides kann gemeinsam verwendet werden: \Npawl{Furia nga lu nitram, lu oe
kop nitram n\`iteng}, \D{Da Du gl\"ucklich bist, bin ich ebenfalls gleichwohl gl\"ucklich}.
% http://naviteri.org/2011/05/weather-part-2-and-a-bit-more-2/comment-page-1/#comment-779


\section{Aspekt und Zeitform}

\subsection{Die Rolle des Kontexts} Im Na’vi sind die Verben oftmals unmarkiert, was die
Zeitform und den Aspekt betrifft, so da\ss{} ein Verb ohne jedweden Infix oder h\"ochstens
eins mit dem subjunktiven Infix \"ubrig bleibt. In Abwesenheit anderer Informationen,
beispielsweise einem Adverb der Zeit oder eine Unterbrechung der Rede, setzt ein
unver\"andertes Verb die Zeit und/oder den Aspekt des Verbs aus dem letzten Satz fort.
\index{Verb!unmarkiert}

\subsubsection{} Auch wenn ein Nebensatz vor einem Hauptsatz stehen kann, \"ubernimmt er
die Zeitform und den Aspekt des Hauptsatzes: \Npawl{Oel foru f\`iayl\`i’ut \uwave{tolìng}
a krr, kxawm oe \uwave{harmah\"angaw}}, \D{Als ich ihm diese Worte \uwave{gesagt hatte},
mu\ss{} ich schlafend \uwave{gewesen sein}} oder \Npawl{T\`i’eyngit oel \uwave{tolel} a
krr, ayngaru \uwave{payeng}}, \D{Sowie ich eine Antwort \uwave{erhalte}, \uwave{werde}
ich es Dich \uwave{wissen lassen}}.

\subsection{Das unmarkierte Verb} Das unmarkeirte Verb hat noch zwei weitere Aufgaben.
Erst einmal kann es den Pr\"asens anzeigen: \Npawl{Ayngaru seiyi irayo}, \D{Ich danke euch}.
Zweitens zeigt es gewohnheitsm\"a\ss{}ige oder allgemeine Aussagen an: \Npawl{Nga za’u
f\`itseng px\`im srak?}, \D{Kommst Du oft hierher?} oder \N{Lu fo lehrrap},
\D{Sie sind gef\"ahrlich}.\index{Verb!Pr\"asens}\index{Verb!unmarkiert}

\subsection{Aspekt} Allgemeinhin zeigt das Na’vi den Aspekt \"ofter an als die Zeitform.
\footnote{Der Asspekt eines Verbs ist eine Schwierigkeit f\"ur Sprecher des Deutschen
oder der meisten anderen europ\"aischen Sprachen, da diese Zeitform und Aspekt in ihren
Verben miteinander vermischen und es so schwierig machen, sie zu unterscheiden. Die
Verwechslungsgefahr, die hier f\"ur Anf\"anger besteht, ist die \"Uberlegung, da\ss{}
der Aspekt sich auf die Abgeschlossenheit oder Nichtabgeschlossenheit einer Handlung
bezieht. Dies ist nicht der Fall. Stattdessen h\"angt der Aspekt damit zusammen, wie ein
Sprecher eine Begebenheit \textit{darzustellen} w\"unscht. Zum Beispiel:

\begin{quotation}
\noindent 1. Ich ging zum Laden. (perfektiv)
2. Als ich zum Laden ging (imperfektiv), sah ich etwas h\"ochst Erstaunliches. (perfektiv)
\end{quotation}
Sowohl in Satz 1 als auch in Satz 2 ist die Handlung, zum Laden zu gehen, in sich
abgeschlossen, nur in Satz 2 dient sie als Bezugspunkt f\"ur die zweite, perfektive,
Aussage.

In komplexen s\"atzen k\"ann der Aspekt den Sinn von Abgeschlossenheit oder
Nichtabgeschlossenheit in Bezug auf andere Glieds\"atze im kompletten Satz
\"ubernehmen, aber das sind spezielle Einsatzgebiete.}
Es ist hilfreich, den Perfektiv als einen bestimmten Zeitpunkt eines Ereignisses
aufzufassen, w\"ahrend der Imperfektiv den Hintergrund definiert: \Npawl{Tengkrr
palulukan moene kxll \uwave{sarmi}, \uwave{poltxe} Neytiril aylì'ut a frakrr ’ok sey\"a
layu oer}, \D{Als der Thanator auf uns beide zust\"urmte, sagte Neytiri etwas, an das
ich mich f\"ur immer erinnern werde}.

\subsection{Gleichzeitiger Imperfektiv} Weil der Imperfektiv eine andauernde Angelegenheit
darstellt, kann er in komplexen S\"atzen eingesetzt werden, um gleichzeitige
Handlungsabl\"aufe darzustellen: \Npawl{F\`itxon yom tengkrr \uwave{teruvon}},
\D{Diese Nacht essen wir, w\"ahrend wir lernen}.\index{Imperfektiv!gleichzeitig}
\LNWiki{14/3/2010}{http://wiki.learnnavi.org/index.php/Canon/2010/March-June\%23A_Collection}

\subsection{Vorzeitiger Perfektiv} In komplexen S\"atzen kann der Perfektiv anzeigen, da\ss{}
die Handlung in einem Nebensatz vor dem Ereignis im Hauptsatz abgeschlossen wurde.
\index{Perfektiv!vorzeitig}

\begin{quotation}
\noindent\Npawl{T\`i’eyngit oel \uwave{tolel} a krr, ayngaru payeng}.\\
\indent\D{Wenn ich eine Antwort \uwave{erhalte}, \uwave{werde} ich es Dich
\uwave{wissen lassen}}.\\

\noindent\Npawl{Fori mawkrra fa renten \uwave{ioi s\"apoli} holum}.\\
\indent\D{Nachdem sie ihre Schutzbrillen \uwave{aufgesetzt hatten}, brachen sie auf}.
\end{quotation}

\subsection{Punktueller Perfektiv} Der Perfektiv wird in einigen Ausdr\"ucken, die nur aus
einem Verb bestehen, verwendet um anzuzeigen, da\ss{} ein Ereignis in einem Augenblick
passiert ist: \N{Tslolam}, \D{Kapiert, verstanden} oder \N{Rolun}, \D{Gefunden!}
Frommer sagt \N{Tolel}, \D{Hab's!}, als Ausdruck f\"ur einen Geistesblitz.
\index{Perfektiv!punktuell}

\subsection{Zeitform} Die Zeitformen des Na’vi definieren, wie in menschlichen Sprachen auch,
den zeitlichen Bezugsrahmen f\"ur ein Ereignis.

\QUAESTIO{Es gibt zu wenig Beispiele komplexer S\"atze, um sich \"uber den relativen
Zeitbezug in Nebens\"atzen sicher sein zu k\"onnen.}

\subsection{Nahe Zeitformen} Die nahe Vergangenheit und Zukunft bezeichnen Ereignisse in
einem Zeitrahmen, in dem "`nah"' keinen absoluten Zeitabstand angibt, sondern durch den
Zusammenhang und die Perspektive des Sprechers bestimmt wird.

\subsection{"`Beabsichtigte Zukunft"'} Die "`beabsichtige Zukunft"', die durch \N{INF{\`isy}}
und \N{\INF{asy}} gebildet werden, zeigt die Absicht des Sprechers an, eine bestimmte
Situation herbeizuf\"uhren, anstatt eine Voraussage \"uber die Zukunft abzugeben:
\Npawl{Ayoe ke \uwave{wasyem}}, \D{Wir werden nicht k\"ampfen} oder \Npawl{Tafral ke
\uwave{l\`isyek} oel ngey\"a keye’ungit}, \D{Deshalb werde ich Deinen Irrsinn nicht befolgen}.
\label{syn:verb:intenfut}\index{Zukunft!beabsichtigt}

\section{Subjunktiv}
\index{Subjunktiv}
\noindent Der Subjunktiv wird im Na’vi viel verwendet. Au\ss{}erhalb seines Einsatzes in
eigenst\"andigen S\"atzen ist der Einsatz des Subjunktivs im Na’vi stark grammatikalisiert,
d. h. seine Verwendung ist in bestimmten grammatikalischen Konstruktionen notwendig, ohne
dabei einen \textit{modus irrealis} anzudeuten.

\subsection{Optativ} Wird verwendet, um einen Wunsch anzuzeigen: \Npawl{Oey\"a swizaw
n\`ingay tivakuk}, \D{La\ss{} meinen Pfeil sein Ziel sicher treffen}.
\index{Subjunktiv!optativ}

\subsection{N\`irangal} Nicht erf\"ullbare W\"unsche werden mit dem Adverb \N{nìrangal}
gebildet, welches f\"ur einen unerf\"ullbaren Wunsch in der Gegenwart vom imperfektiven
und f\"ur einen unerf\"ullbaren Wunsch in der Vergangenheit mit dem perfektiven Subjunktiv
gebildet. Das kann im Deutschen mit "`Wenn doch nur"' respektive "`Ich w\"unschte, da\ss{}"'
ausgedr\"uckt werden.
\N{N\`irangal lirvu oeyä frrnenur lora sanh\`i}, \D{Ich w\"unschte, da\ss{} meine Kinder
h\"ubsche Sterne h\"atten} oder \N{N\`irangal oel tslilvam n\`i’ul}, \D{Wenn ich doch nur
mehr verstanden h\"atte}.
\index{niìrangal@\textbf{n\`irangal}}
\LNWiki{14/3/2010}{http://wiki.learnnavi.org/index.php/Canon/2010/March-June\%23A_Collection}

\subsection{Modales Komplement} Das komplement\"are Verb eines Modalverbs (z. B. \N{zene},
\D{m\"ussen}, \N{tsun}, \D{k\"onnen}, usw. steht im Subjunktiv, wie in \N{ayngari zene
hivum}, \D{Ihr m\"u\ss{}t gehen} oder \Npawl{Oe new n\`itxan ayngaru fyawiv\`intxu},
\D{Ich m\"ochte euch sehr gerne anleiten/f\"uhren} oder \Npawl{Fmawn a tsun oe ayngaru
tiv\`ing}, \D{Neuigkeiten, die ich euch geben kann}.\label{syn:modals}

\subsubsection{} Das von einem Modalverb kontrollierte Verb wird nicht durch Zeit- oder
Aspektinfixe ver\"andert\footnote{Das kontrollierte Verb beh\"alt einen etwaigen reflexiven
oder kausativen Infix.}, sondern lediglich den Subjunktiv. Die Angabe der Zeitform und des
Aspektes kommen ins Modalverb: \N{Oe namew tse’a}, \D{Ich wollte sehen}, niemals
*\N{Oe new tsimve’a}.

\subsubsection{} Bekannte Modalverben und Verben mit modalem Satzbau:
\footnote{\QUAESTIO{Andere Kandidaten: \N{sto} \D{verweigern},
\N{fl\"a} \D{gelingen}, \N{hawl} \D{vorbereiten}.}}
\begin{center}
\begin{tabular}{ll}
\N{fmi} & versuchen \\
\N{ftang} & aufh\"oren \\
\N{kan} & vorhaben \\
\N{may’} & ausprobieren, testen \\
\N{new} & m\"ochten, wollen\footnotemark{} \\
\end{tabular}
\hskip 3em
\begin{tabular}{ll}
\N{sng\"a’i} & beginnen, anfangen \\
\N{tsun} & k\"onnen, in der Lage sein \\
\N{var} & fortsetzen \index{var@\textbf{var}} \\
\N{zene} & m\"ussen \\
\N{zenke} & nicht d\"urfen \\
\end{tabular}
\end{center}
\footnotetext{See also \horenref{syn:modal:new}.}
\index{fmi@\textbf{fmi}!modal}\index{ftang@\textbf{ftang}!modal}
\index{new@\textbf{new}!modal}\index{kan@\textbf{kan}!modal}
\index{snaä'i@\textbf{sng\"a’i}!modal}\index{tsun@\textbf{tsun}!modal}
\index{var@\textbf{var}!modal}\index{zene@\textbf{zene}!modal}
\index{zenke@\textbf{zenke}!modal}\index{may'@\textbf{may’}!modal}

\noindent\NTeri{25/5/2011}{http://naviteri.org/2011/05/some-miscellaneous-vocabulary/}
\LNWiki{1/12/2010}{http://wiki.learnnavi.org/index.php/Canon/2010/October-December\%23As_X_as_Y.3B_Keep_on_keepin.27_on}
\LNWiki{2/2/2011}{http://wiki.learnnavi.org/index.php/Canon/2011/January-March\%23Stop.21}
\Ultxa{2/10/2010}{http://wiki.learnnavi.org/index.php/Canon/2010/UltxaAyharyu\%C3\%A4\%23.C3.ACm.C3.ACy_and_modal_kan}

\subsubsection{} Beachten Sie, da\ss{} Modalverben als intransitiv gelten, wobei das Subjekt
des Modalsatzes unabh\"angig von der Transitivit\"at des kontrollierten Verbs im Nominativ
steht.
\N{\uwave{Oe} new yivom teylut}, \D{Ich m\"ochte Teylu essen}.
Aber siehe auch "`Effekte der Wortreihenfolge"', \horenref{pragma:word-order-effects:modals},
f\"ur einige Ausnahmen.

\subsection{New} Zus\"atzlich zum einfachen modalen Gebrauch (siehe oben)
(\horenref{syn:modals}) kann \N{new}, \D{wollen}, einen Nebensatz mit einem anderen Subjekt
als dem des Hauptsatzes mit \N{new}. In diesem Aufbau ist das Verb transitiv, und der
Nebensatz wird mittels \N{a f\`i’ut} respektive \N{futa} angeh\"angt
(\horenref{syn:clause-nom}) und verwendet den Subjunktiv.
\label{syn:modal:new}\index{new@\textbf{new}}
\LNWiki{20/1/2010}{http://wiki.learnnavi.org/index.php/Canon\%23Extracts_from_various_emails}

\begin{quotation}
\noindent \N{Oel new futa po kiv\"a}, \D{Ich will, da\ss{} er geht}. \\
\noindent \N{Ngal tslivam a f\`i’ut new oel}, \D{Ich m\"ochte, da\ss{} Du verstehst}.
\end{quotation}

\subsubsection{} Der Kausativ des transitiven \N{new} erh\"alt einen Nebensatz, der mit
\D{futa} eingeleitet wird: \Npawl{Pol oeru neykew futa oel yivom teyluti}, \D{Er brachte
mich dazu, Teylu essen zu wollen} (w\"ortlich: Er machte mich wollen, da\ss{} ich Teylu esse).
\index{new@\textbf{new}!kausativ}

\subsubsection{} Der modale Gebrauch von \N{kan}, \D{zielen}, folgt der gleichen
Satzstruktur: \N{Oe kan kiv\"a}, \D{Ich gedenke zu gehen} und \N{oel kan futa po kiv\"a},
\D{Ich beabsichtige, da\ss{} er geht}.\index{kan@\textbf{kan}}

\subsection{Andere Verwendungen} Der Subjunktiv wird ebenfalls in S\"atzen, die einen Zweck
angeben, zusammen mit dem Verb \N{fte} (\horenref{syn:purpose}), bedingten S\"atzen
(\horenref{syn:conditionals}) und zusammen mit der Konjunktion \N{tsn\`i} verwendet
\QUAESTIO{wenn zusammen mit bestimmten Verben verwendet} (\horenref{syn:tsni}).


\section{Partizipien und das Gerundium}

\subsection{Partizipien} Der Gebrauch von Partizipien im Na’vi ist eingeschr\"ankt. Sie
d\"urfen nur attributiv eingesetzt werden, niemals pr\"adikativ. Da es sich um Adjektive
handelt, werden sie mit dem Substantiv mittels des Affixes \N{-a-} verbunden
(\horenref{morph:adj-attr}),
\Npawl{\uwave{Palulukan atusaron} lu lehrrap}, \D{Ein \uwave{jagender Thanator} ist
gef\"ahrlich}.\label{syn:part:attr}

\subsubsection{} Einige abgeleitete W\"orter enthalten Partizipien, und diese k\"onnen
pr\"adikativ eingesetzt werden, wie in \Npawl{Lu nga txantslusam}, \D{Du bist weise},
wenn das Partizip Aktiv \N{tsl\INF{us}am} darin enthalten ist.

\subsubsection{} Die Partizipien von mit \N{si} zusammengesetzten Verben z\"ahlen als
ein Wort. Diese werden mit Bindestrich zwischen dem \N{si} und dem zugeh\"origen Wort
geschrieben, wobei das attributive \N{-a-} an die komplette Konstruktion und nicht nur
das \N{si} geh\"angt wird:\label{syn:participle:si-const}
\index{si-Konstruktion@\textbf{si}-Konstruktion!Partizip}

\begin{quotation}
\noindent\N{srung-susi\uwave{a} tute}\\
\noindent\N{tute \uwave{a}srung-susi}
\end{quotation}

\noindent Beide S\"atze bedeuten \D{eine helfende Person}.

\subsection{Gerundien} Jedes Verb kann ohne Einschr\"ankungen in ein Gerundium umgewandelt
werden, das die Aktion des Verbs beschreibt (\horenref{lingop:gerund}). Sie k\"onnen
zusammen mit Adverben verwendet werden (\horenref{syn:adverbs:gerund}), aber sie d\"urfen
keine Subjekte oder direkte Objekte haben: \Npawl{T\`iyusom ’o’ lu}, \D{Essen macht
Spa\ss{}}.\label{syn:gerund}
\LNWiki{18/6/2010}{http://wiki.learnnavi.org/index.php/Canon/2010/March-June\%23Fwa_with_adpositions}

\subsubsection{} Im Deutschen verwandelt sich diese Konstruktion in eine
Infinitivkonstruktion um: "`running a marathon is difficult"' (oder
im Deutschen "`Einen Marathon zu laufen ist schwierig."'). Im Na'vi werden solche
Substantivierungen mit \N{f\`i’u} oder \N{tsa’u} bewerkstelligt (\horenref{syn:clause-nom}):
\Npawl{\uwave{Fwa yom teylut} 'o' lu}, \D{\uwave{Teylu zu essen} macht Spa\ss{}}.
\index{Gerundium!Einsatzgebiete}
\Ultxa{3/10/2010}{http://wiki.learnnavi.org/index.php/Canon/2010/UltxaAyharyu\%C3\%A4\%23Gerunds_vs._Fwa}


\section{Reflexive}
\index{Reflexive!Satzbau}
\subsection{Echte Reflexive} 
Der reflexive Infix \N{\INF{\"ap}} zeigt an, da\ss{} das Subjekt des Verbs eine Handlung
an sich selbst durchf\"uhrt. Das Subjekt steht im Nominativ, nicht im Agens, wie in
\Npawl{Oe ts\"ape’a}, \D{Ich sehe mich selbst}.
\LNWiki{1/2/2010}{http://wiki.learnnavi.org/index.php/Canon\%23Reflexives_and_Naming}

\subsection{Intransitive Reflexive} Zusammen mit intransitiven Verben, die Dativobjekte
haben, wird \N{sno} verwendet.\index{Reflexiv!intransitiv}

\begin{quotation}
\noindent\N{Po yawne lu snor}, \D{Er liebt sich selbst.}
\end{quotation} \index{sno@\textbf{sno}!mit intransitiven Reflexiven}
\noindent \NTeri{31/12/2011}{http://naviteri.org/2011/12/one-more-for-2011/}

\subsection{Detransitiv} Der Reflexivinfix kann auch verwendet werden, um intransitive
Verben zu erzeugen\footnote{Sch\"uler romanischer Sprachen wird dies vertraut vorkommen:
\textit{Je me lave} gegen\"uber \textit{Je lave ma voiture}.}, so wie \N{win s\"api},
\D{sich beeilen}\footnote{Im Gegensatz zum Englischen, wo der reflexive Bezug oftmals
verlorengeht, bleibt er im Deutschen erhalten}.

\subsection{Wechselseitigkeit} Erscheint ein reflexives Verb zusammen mit dem Adverb
\N{f\`itsap}, \D{gegenseitig}, dr\"uckt es Wechselseitigkeit aus: \Npawl{Mefo f\`itsap
m\"apoleyam tengkrr tsngawv\`ik}, \D{Die beiden umarmten sich gegenseitig und weinten}.
\NTeri{30/10/2011}{http://naviteri.org/2011/10/more-vocabulary-a-bit-of-grammar/}
\index{fiìtsap@\textbf{f\`itsap}}\index{Wechselseitigkeit}

\subsubsection{} Es gibt zwei M\"oglichkeiten f\"ur intransitive Verben, die
Dativobjekte haben:

\begin{quotation}
\noindent\Npawl{Moe smon moeru f\`itsap.} \D{Wir (beide) kennen uns (gegenseitig).} \\
\noindent\Npawl{Moe smon f\`itsap.} \D{Wir (beide) kennen uns (gegenseitig).}
\end{quotation}

\noindent Reflexive der dritten Person mit beliebigem Numerus verwenden den Dativ
von \N{sno}.

\begin{quotation}
\noindent\Npawl{Fo smon (snoru) f\`itsap n\`iwotx.}, \D{Sie kennen sich alle
gegenseitig}.
\end{quotation}

\noindent\NTeri{31/12/2011}{http://naviteri.org/2011/12/one-more-for-2011/}

\section{Kausativ}
\noindent The causative infix \N{\INF{eyk}} increases the transitivity
of a verb, adding another argument.  All causative verbs are thus
transitive, requiring the agentive case for the subject.
\label{syn:causative}

\subsection{Kausativ intransitiver Verben} Wird ein intransitives Verb kausativ,
so steht der "`Verursachte"', der im Nominativ gestanden hatte, nachher
im Patiens.\index{Kausativ!intransitiver Verben}

\begin{quotation}
\noindent\N{\uwave{Oe} kol\"a neto}, \D{Ich ging fort}.\\
\noindent\N{Pol \uwave{oeti} keykol\"a neto}, \D{Sie brachte mich dazu zu gehen}.
\end{quotation}

\subsection{Kausativ transitiver Verben} Wird ein transitives Verb kausativ, steht
der "`Verursachte"', der vorher im Agens gestanden hatte, im Dativ. Dies
bel\"a\ss{}t den urspr\"unglichen Patiens unangetastet.
\index{Kausativ!transitiver Verben}\label{syn:trans-causative}
\index{Fall!Dativ!mit Kausativ}

\begin{quotation}
\noindent\N{\uwave{Neytiril} \uuline{yerikit} tolaron}, \D{Neytiri hat einen
Hexapeden gejagt}.\\
\noindent\Npawl{Eytukan\`il \uwave{Neytirir} \uuline{yerikit} teykolaron}\\
\indent \D{Eytukan lie\ss{} Neytiri einen Hexapeden jagen}. 
\end{quotation}

\subsubsection{} Der "`Verursachte"' kann ebenfalls durch die Adposition \N{fa},
\D{von, durch, mittels}, angezeigt werden. Dies entzieht dem "`Verursachten"' den
Fokus in gewisser Weise und plaziert ihn stattdessen entweder auf den Verursacher
oder das Objekt.\index{fa@\textbf{fa}!mit Kausativ}

\begin{quotation}
\noindent\N{\uwave{Neytiril} \uuline{yerikit} tolaron}, \D{Neytiri hat einen
Hexapeden gejagt}.\\
\noindent\Npawl{Eytukan\`il \uwave{fa Neytiri} \uuline{yerikit} teykolaron}, \\
\indent \D{Eytukan lie\ss{} einen Hexapeden von Neytiri jagen}.
\end{quotation}



\section{Ambitransitivit\"at}
\noindent Ein an sich transitives Verb kann ein Subjekt im Nominativ statt dem
Agens haben, wenn das direkte Objekt als irrelevant angesehen werden kann und
lediglich die Aktion z\"ahlt. So ist beispielsweise \N{Oe taron}, \D{Ich jage},
eine allgemeine Aussage \"uber jemandes Aktivit\"aten, in der das, was man
jagt, nicht von Interesse ist.\index{Antipassiv}\index{Ambitransitivit\"at}
\NTeri{28/3/2012}{http://naviteri.org/2012/03/spring-vocabulary-part-1/}

\begin{quotation}
\noindent\Npawl{Ngal pelun faystxenut frakrr tsyär?}\\
\indent\D{Warum lehnst du st\"andig diese Gaben ab?} \hskip3em gegen\"uber \\
\noindent\Npawl{Nga pelun frakrr tsyär?}\\
\indent\D{Warum lehnst du st\"andig alles ab?} oder
\indent\D{Warum lehnst du st\"andig solche Sachen ab?}
\end{quotation}

\noindent Man kann dieses Wechselmuster auch als "`Antipassiv"’-Konstruktion
bezeichnen, die im Na’vi ohne Einschr\"ankungen verwendet werden kann.

\subsection{Ausgelassenes Objekt} Diese Verwendung sollte von der Auslassung
eines direkten Objektes, die im Kontext existiert, unterschieden werden,
beispielsweise:

\begin{quotation}
\noindent\N{Ngal ke tse’a txepit srak?} \D{Siehst Du das Feuer nicht?}\\
\noindent\N{Oel tse’a.} \D{Ich sehe (es).}
\end{quotation}

\noindent Hier wird das direkte Objekt einfach nicht erw\"ahnt, anstatt es
komplett zu unterdr\"ucken, so da\ss{} Verb und Subjekt der normalen transitiven
Satzkonstruktion folgen.\index{Direktes Objekt!ausgelassen}

\subsection{Kausativ} Es gibt keine M\"oglichkeit, den Antipassiv im Kausativ zu
erkennen. Beispielsweise kann die Aktion des Satzes \N{Oel poru teykaron},
\D{Ich veranlasse ihn zu jagen} entweder \N{Po taron}, \D{Er jagt (irgendetwas, das
nicht interessiert)} oder \N{Pol taron}, \D{Er jagt (etwas Bestimmtes)} sein.
\index{Antipassiv!Kausativ}
\Ultxa{3/10/2010}{http://wiki.learnnavi.org/index.php/Canon/2010/UltxaAyharyu\%C3\%A4\%23Causative_for_ambitransitive_verbs}


\section{Befehle}
\index{Befehle}\index{Imperativ}
\subsection{nicht markiert} Befehle im Na’vi bed\"urfen keines besonderen Infixes.
Positive Befehle verwenden einfach den Verbstamm: \N{Kä! Kä!} \D{Geh! Geh!}, 
\Nfilm{Mefoti y\`im}, \D{Bindet sie!} Das Pronomen kann ebenfalls explizit
angegeben werden: \Npawl{’Awpot set ftxey ayngal}, \D{Du w\"ahlst jetzt eins aus}.
\index{Befehl!einfach}

\subsection{mit dem Subjunktiv} Ein Befehl kann auch den subjunktiven Infix \N{\INF{iv}}
verwenden. Frommer erkl\"art: "`Zu einem fr\"uheren Zeitpunkt in der Entwicklung der
Sprache gab es wahrscheinlich eine h\"ofliche/vertraute Form (mit \N{\INF{iv}} als
der h\"oflicheren Form), aber das ist nicht mehr der Fall. Sie k\"onnen beide
nach Belieben eingesetzt werden. `Geh' kann man sowohl als \N{k\"a} als
auch als \N{kiv\"a} ausdr\"ucken."'
\index{Subjunktiv!mit Befehlen}\index{Befehl!mit Subjunktiv}

\subsection{Verbote} Negative Befehle werden nicht mit dem negativen Adverb \N{ke}
negiert, sondern verwenden das Wort \N{r\"a’\"a}, wie in \N{R\"a’\"a hahaw},
\D{Schlaf nicht}.
\label{syntax:prohibitions}\index{Befehl!negativ}\index{Verbot}

\subsubsection{} In mit \N{si} konstruierten Verben schiebt sich das \N{r\"a’\"a} zwischen
das Substantiv und das \N{si}: \N{Txopu r\"a’\"a si} \D{Hab keine Angst} oder
\N{Tsakem r\"a’\"a sivi}, \D{Tu das nicht} (siehe auch \horenref{syn:neg:si-const}).
\index{si-Konstruktion@\textbf{si}-Konstruktion!Verbot}

\section{Fragen}
\index{Frage}
\subsection{Ja-Nein-Fragen} Einfache Ja-Nein-Fragen werden mit dem Partikel \N{srak(e)}
markiert. Taucht es am Ende auf, ist es normalerweise \N{srak}, das l\"angere \N{srake}
erscheint an anderen Stellen im Satz: \Npawl{Ngaru lu fpom srak?}, \D{Geht es Dir gut?}

\subsection{Ftxey... Fuke} Zus\"atzlich zu \N{srak(e)} kann eine Ja-Nein-Frage auch mit
einer Redewendung mit \N{ftxey}, \D{w\"ahlen}, und \N{fu\ACC{ke}}, \D{oder nicht},
aufgebaut werden. So k\"onnen Sie entweder \Npawl{Srake nga za’u}, \D{Kommst Du} oder
\Npawl{Ftxey nga za’u fuke}, \D{Kommst DU oder nicht?}
\index{ftxey@\textbf{ftxey}}\index{fuke@\textbf{fuke}}
\index{Frage!direkt mit \textbf{ftxey... fuke}}\label{syn:question:ftxey}
\LNWiki{24/3/2010}{http://wiki.learnnavi.org/index.php/Canon/2010/March-June\%23If_and_Whether}

\subsection{W-Fragen} Die Benutzung eines Fragewortes, das \N{pe+} enth\"alt, ist
ausreichend, um eine Frage zu erzeugen: \N{Kempe si nga?}, \D{Was tust Du?}
In vielen Sprachen mu\ss{} das Fragewort an erster Stelle im Satz stehen, wogegen das
Na’vi keine solche Bedingung stellt: \Nfilm{F\`iswir\"ati ngal \uwave{pelun} molunge
f\`itsenge}, \D{\uwave{Warum} hast Du diese Kreatur hergebracht?}
\index{Frage!mit Fragewort}

\subsection{Frage mit Anh\"angsel} Die Frage mit Anh\"angsel im Na’vi (Englisch: "`right?"',
Franz\"osisch: "`n’est-ce pas?"', Deutsch: "`stimmt’s?"') wird entweder mit \N{kefya srak}
oder einfach \N{kefyak} (abgeleitet von \N{fe f\`ifya srak}) markiert.
\index{Frage!Anh\"angsel}\index{kefyak, kefya srak@\textbf{kefyak, kefya srak}}
\LNWiki{1/3/2010}{http://wiki.learnnavi.org/index.php/Canon\%23Tag_Question}

\subsection{Mutma\ss{}ende Fragen} Fragen, zu denen der Fragesteller nicht erwartet, da\ss{}
die Zuh\"orer die Antwort kennen, werden mit dem Mutma\ss{}ungsinfix \N{\INF{ats}} markiert:
\Npawl{Pol pesenget tatsok}, \D{Wo zum Henker k\"onnte sie sein?} oder \Npawl{Srake pxefo
li pol\"ahatsem}, \D{Ich frage mich, ob die Drei schon angekommen sind.}
\NTeri{30/10/2011}{http://naviteri.org/2011/10/more-vocabulary-a-bit-of-grammar/}

\section{Verneinung}

\subsection{Einfache Verneinung} Das Adverb \N{ke} wird verwendet, um einen Satz zu
verneinen: \Npawl{F\`itxon na ton alahe n\`iwotx pelun ke lu teng},
\D{Warum ist diese Nacht nicht wie alle anderen N\"achte?}
\index{Verneinung}\index{ke@\textbf{ke}}

\subsubsection{} In mit \N{si} konstruierten Verben steht das \N{ke} vor dem \N{si}, wie in
\N{Po pamrel ke si}, \D{Er schreibt nicht}. Die Betonung verschiebt sich dabei vom
Substantiv oder Adjektiv zum \N{ke}: \N{Pamrel \ACC{ke} si} (siehe auch
\horenref{syntax:prohibitions}). \label{syn:neg:si-const}
\index{si-Konstruktion@\textbf{si}-Konstruktion!Verneinung}
\index{Verneinung!si-Konstruktion@\textbf{si}-Konstruktion}
\index{ke@\textbf{ke}!mit \textbf{si}-Konstruktion}

\subsubsection{} Imperative werden mit dem Adverb \N{r\"a’\"a} verneint.
Siehe auch \horenref{syntax:prohibitions}.\index{Verbote}

\subsection{Gedoppelte Verneinung} Wird ein negatives Adverb, Pronomen oder Substantiv
verwendet, ist das \N{ke} f\"ur das Verb dennoch erforderlich: \Npawl{Ke’u ke lu ngay},
\D{Nichts stimmt} oder \N{Sl\"a ke st\"a'n\`i kawkrr}, \D{Aber er f\"angt (sie) nie}.
\index{Verneinung!gedoppelt}\label{syn:neg:pleon}
\LNWiki{2/5/2010}{http://wiki.learnnavi.org/index.php/Canon/2010/March-June\%23Double_Negatives_Required}

\subsubsection{} Wird das Pronomen \N{fra-} negiert, so ist es das Verb auch:
\Npawl{Ke frapo ke tslolam}, \D{Nicht jeder hat verstanden}.
\index{fra-@\textbf{fra-}!mit \textbf{ke}}
\Ultxa{3/10/2010}{http://wiki.learnnavi.org/index.php/Canon/2010/UltxaAyharyu\%C3\%A4\%23Ke_with_fra-}

\subsection{Kaw’it} Ein Wort oder Gliedsatz kann zur Verneinung
mittels \N{ke\dots kaw’it},
\D{\"uberhaupt nicht}, herausgegriffen werden, wie in \N{Do ke lu ’ewan kaw’it},
\D{Sie sind \"uberhaupt nicht jung}.
\index{kaw'it@\textbf{kaw’it}}
\LNWiki{6/4/2010}{http://wiki.learnnavi.org/index.php/Canon/2010/March-June\%23April_6_Miscellany}


\section{Komplexe S\"atze}

\subsection{Zeitformen und Aspekte in abh\"angigen Subjunktiven} \QUAESTIO{Teilen
abh\"angige Verben Zeitform, Aspekt und Stimmungslage mit den Verben,
von denen sie kontrolliert werden?}

\subsection{Absicht} Absichtss\"atze bekommen die Konjunktion \N{fte} (verneint \N{fteke})
zusammen mit dem Subjunktiv: \N{Sawtute zera’u fte fol Kelutralti skiva’a},
\D{Die Himmelsmenschen kommen, um den Heimatbaum zu zerst\"oren}.
\label{syn:purpose}\index{Absichtssatz}
\index{fte@\textbf{fte}}\index{fteke@\textbf{fteke}}

\subsubsection{} Absichtss\"atze werden im Na’vi in verschiedenen Situationen eingesetzt,
in denen im Deutschen eine Infinitivkonstruktion verwendet wird: \Npawl{Pxiset ke lu oeru
krr \uwave{fte} t\`iu’eyngit \uwave{tiv\`ing}} \D{Momentan habe ich nicht die Zeit, um
eine Antwort zu geben}.

\subsection{Asyndeton} Kurze, gleichf\"ormige S\"atze\footnote{sprich: S\"atze, die dem
gleichen grammatikalischen Aufbau folgen.} k\"onnen ohne Konjunktion aneinander
geh\"angt werden. \Npawl{Yola krr, txana krr, ke tsranten}, \D{Es spielt keine Rolle,
wie lange es dauert} oder w\"ortlich: \D{Kurz, lang, ist nicht wichtig}.
\Npawl{’Uo a fpi rey’eng \uwave{Eywa’evengmì ’Rrtam\`i} tsranten n\`itxan awngaru n\`iwotx},
\D{Etwas, das jedem von uns zum Wohle des Gleichgewichts des Lebens wichtig ist,
\uwave{sowohl auf Pandora als auch auf der Erde}} oder \Npawl{Lora ayl\`i’u, lora ays\"afp\`il},
\D{Sch\"one Worte, sch\"one Ideen}.
\index{Asyndeton}\index{Konjunktionen!ausgelassen}

\subsubsection{} Zwei Verben ohne Konjunktion in Folge geben eine Abfolge an:
\Npawl{Za’u kaltx\`i si ko}, \D{Komm (und dann) sage Hallo.}


\section{Relativs\"atze und Gliedsatzattribution}
\index{Relativsatz}
\subsection{Partikel "`a"'} Relativs\"atze werden im Na’vi mit dem Partikel \N{a} gebildet.
\index{a@\textbf{a}}\label{syn:a} \"Ahnlich der adjektivischen Attribution kann ein
Relativsatz dem Wort, das es ver\"andert, voran- oder nachgestellt sein: \Npawl{\uwave{Po
tsane karm\"a a tsengit} ke ts\`ime’a oel}, \D{Ich habe \uwave{die Stelle, zu der er
gegangen ist}, nicht gesehen}\footnote{Zwischen einem nachgestellten Relativsatz und dem
Wort, das er ver\"andert, kann im Deutschen auch das Verb des \"ubergeordneten Satzes stehen}
oder \Npawl{\uwave{Palulukan a teraron} lu lehrrap}, \D{\uwave{Ein Thanator, der jagt}, ist
gef\"ahrlich}.

\subsubsection{} Beachten Sie, da\ss{} das attributive \N{a} ein Partikel und kein
Pronomen ist und daher keine Fallendungen bekommt.

\subsection{Verweishierarchie} Wenn der Kopf\footnote{Der "`Kopf"' des Relativsatzes ist
das Substantiv, dem der Relativsatz zugeordnet ist. Es \"ubernimmt sowohl im Haupt- als
auch im Nebensatz eine syntaktische Rolle. Beispielsweise ist in dem Satz \D{Ich sehe
den Mann, der rennt} das Wort "`Mann"’ das direkte (oder Akkusativ-)Objekt des Hauptsatzes
"`Ich sehe den Mann"’, aber gleichzeitig auch das Subjekt des Relativsatzes "`Der Mann
rennt."’ Ein Element, das beiden Glieds\"atzten gemeinsam ist, wird manchmal auch als
"`Angelpunkt"’ bezeichnet.} eines Relativsatzes dessen Subjekt
oder direktes Objekt ist, wird er ausgelassen.

\begin{quotation}
\noindent \N{\uwave{Ngal tse’a a tute} lu eyktan}.
  \D{\uwave{Der Mann, den Du siehst}, ist Anf\"uhrer}.
\noindent \N{\uwave{Ngati tse’a a tute} lu eyktan}.
  \D{\uwave{Der Mann, der Dich sieht}, ist Anf\"uhrer}.
\end{quotation}
\noindent F\"ur andere Adpositionals\"atze mu\ss{} ein wiederaufnehmendes Pronomen verwendet
werden --- \N{po} f\"ur belebte K\"opfe und \N{tsaw} f\"ur Unbelebtes.
% Feb 18: http://wiki.learnnavi.org/index.php/Canon#More_extracts_from_various_emails

\begin{quotation}
\noindent \Npawl{Poru mesyal lu a ikran}, \D{Ein Ikran mit zwei Fl\"ugeln}\\
\noindent \Npawl{Po \uwave{tsane} karm\"a a tsengit ke ts\`ime’a oel}.\\
\indent\D{Ich habe die Stelle nicht gesehen, \uwave{zu der (ihr)} sie gegangen ist}.
\noindent \N{F\`ipo lu tute a oe \uwave{pohu} per\"angkxo.} \\
\indent\D{Dies ist die Person, \uwave{mit der (ihr)} ich spreche}.
\end{quotation}

\subsubsection{} Wenn der Kopf eines Relativsatzes dessen direktes Objekt ist, dann mu\ss{}
das Subjekt des Verbs dennoch mit dem Suffix f\"ur den Agens markiert werden, wie in dem
obigen Satz \N{\uwave{ngal} tse’a a tute}, \D{der Mann, den du siehst}, nicht
*\N{\uwave{nga} tse’a a tute} oder \Npawl{Teylu a \uwave{oel} yerom lu ftx\`ilor},
\D{Die Teylu, die ich esse, sind k\"ostlich}.
\NTeri{28/3/2012}{http://naviteri.org/2012/03/spring-vocabulary-part-1/}

\subsection{Andere Attributivs\"atze} Anstatt wie im Deutschen Substantive direkt mit
Pr\"apositionals\"atzen zu ver\"andern ("`Der Mann im Mond"'), h\"angt das Na’vi solche
S\"atze mittels \N{a} an Substantive an, wie in \Nfilm{F\`ipo lu \uwave{vrrtep a m\`i
sokx atsleng}}, \D{Dies ist ein \uwave{D\"amon in einem falschen K\"orper}} oder
\N{Ngey\"a \uwave{teri faytele a ays\"anumeri}},
\D{Deine \uwave{Anweisungen bez\"uglich dieser Angelegenheiten}}.
\index{Adposition!Attributivsatz}

\subsubsection{} Farbschattierungen k\"onnen mittels der Adposition \N{na},
\D{\"ahnlich, so wie} pr\"azisiert werden (\horenref{syn:adp:na}). Eine solche
Konstruktion attributiv zu verwenden, wird der komplette Gliedsatz mit Bindestrichen
versehen und als normales Adjektiv behandelt.
Beispielsweise \Npawl{Ean na ta’leng}, \D{(Na’vi-)hautblau}:
\begin{quotation}
\indent\N{F\`isyulang \uwave{aean-na-ta’leng} lor lu n\`itxan.}\\
\indent\N{F\`isyulang \uwave{ata’lengna-ean} lor lu n\`itxan.}\\
\indent\N{\uwave{Ean-na-ta’lenga} f\`isyulang lor lu n\`itxan.}\\
\indent\N{\uwave{Ta’lengna-eana} f\`isyulang lor lu n\`itxan.}
\end{quotation}
\index{na@\textbf{na}!mit Farben}\label{syn:attr:na}

\subsubsection{} Einzelne Adverben k\"onnen ebenfalls attributiv verwendet werden:
\Npawl{Ke zasyup l\`i’Ona ne kxutu a m\`ifa fu a wrrpa}, \D{Die L\`i’Ona werden weder
dem \"au\ss{}eren noch dem inneren Feind anheimfallen}.
\index{Adverben!attributiv}


\subsection{Gliedsatzsubstantivierung} Vollst\"andige Glieds\"atze k\"onnen in
Substantive umgewandelt und in den Satzbau mittels des Attributivpartikels in den
Satzbau eingef\"ugt werden, wobei der Gliedsatz entweder mit \N{f\`i’u} oder \N{tsa’u}
im Hauptsatz verankert wird. Dies geschieht oft genug, so da\ss{} das Pronomen und
das Partikel verschmelzen (siehe auch \horenref{morph:fwa-tsawa}). \label{syn:clause-nom}
\index{fwa@\textbf{fwa}!Anwendung}
\index{fula@\textbf{fula}!Anwendung}
\index{futa@\textbf{futa}!Anwendung}
\index{furia@\textbf{furia}!Anwendung}

\subsubsection{} Wie auch in einem Relativsatz wird das Bindungspronomen dekliniert,
um seiner Rolle im Hauptsatz zu entsprechen. Beispielsweise mit \N{fwa} im Nominativ als
intransitives Subjekt f\"ur \N{lu}:

\begin{quotation}
\noindent\Npawl{Law lu oeru \uwave{fwa nga m\`i reltseo nolume n\`itxan}}.\\
\indent\D{Es ist mir klar, \uwave{da\ss{} Du auf dem Bereich der Kunst viel gelernt hast}}.
\end{quotation}

\noindent Das themenbezogene \N{a f\`i’uri} zusammen mit \N{irayo si}:
\begin{quotation}
\noindent\Npawl{\uwave{Ngal oey\"a ’upxaret aysuteru fpole’ a f\`i’uri}, ngaru irayo seiyi
oe n\`itxan.}\\
\indent\D{Ich danke Dir sehr, \uwave{da\ss{} Du dem Volk meine Nachricht \"uberbracht hast}}.
\end{quotation}

\noindent Mit \N{futa} als direktem Objekt des Verbs \N{omum}:
\begin{quotation}
\noindent\Npawl{Ulte omum oel \uwave{futa t\`ifyaw\`intxuri oey\"a perey aynga n\`iwotx}.}\\
\indent\D{Und ich wei\ss{}, \uwave{da\ss{} ihr alle auf meine Anleitung wartet}}.
\end{quotation}

\subsubsection{} Bestimmte Verben ben\"otigen sehr oft eine bestimmte Satzsubstantivierung.
So brauchen beispielsweise Nebens\"atze mit \N{omum}, \N{wissen}, einen Akkusativsatz
(i. d. R. mit \N{futa} oder \N{a f\`i’ut}).

\subsubsection{} Glieds\"atze k\"onnen auch mit Varianten von \N{tsa’u} substantiviert
werden. Der Unterschied zwischen \N{f\`i’u} und \N{tsa’u} besteht darin, da\ss{} \N{tsa’u}
dann angewendet wird, wenn der Gliedsatz, den es verankert, sich auf etwas Altes in einer
Rede bezieht, etwas, das zuvor behandelt worden ist. Diese Feinheit ist jedoch nicht
notwendig, und Varianten von \N{f\`i’u} sind niemals verkehrt.
\QUAESTIO{Beispielunterhaltung, die beides verwendet?}
\index{tsawa@\textbf{tsawa}!Anwendung}
\index{tsata@\textbf{tsata}!Anwendung}
\index{tsaria@\textbf{tsaria}!Anwendung}
\LNWiki{18/6/2010}{http://wiki.learnnavi.org/index.php/Canon/2010/March-June\%23The_contrast_between_fwa.2Ftsawa.2C_furia.2Ftsaria}

\subsubsection{} Das Substantiv \N{t\`ikin}, \D{Notwendigkeit}, wird zusammen mit einem
Attributivsatz f\"ur den Ausdruck "`ben\"otigen"' verwendet: \Npawl{Awngaru lu t\`ikin
a nume n\`i’ul}, \D{Wir m\"ussen mehr lernen} (w\"ortlich: "`Wir haben die Notwendigkeit,
mehr zu lernen"').\index{tìkin@\textbf{t\`ikin}!mit Attributivsatz}
Es kann auch unpers\"onlich verwendet werden: \N{Lu t\`ikin a \dots}, \D{Es besteht die
Notwendigkeit f\"ur \dots}


\subsection{Substantivierte Glieds\"atze mit Adpositionen} Substantivierte Glieds\"atze
k\"onnen mit bestimmten Adpositionen verwendet werden und ergeben eine Bedeutung, die
im Deutschen bestimmter Konjunktionen und Infinitivkonstruktionen entspricht.
\N{Oe ke tsun stivawm fayfnel\`i’ut \uwave{luke fwa sng\"a’i tsngivawv\`ik}},
\D{Ich kann solche Worte nicht anh\"oren, \uwave{ohne anzufangen zu weinen}}.
\label{syn:rel:nom-adp}
\index{Adpositionen!mit Gliedsatzsubstantivierungen}
% http://forum.learnnavi.org/language-updates/txelanit-hivawl/


\subsubsection{} \QUAESTIO{Eine Liste g\"ultiger Adpositionen w\"are gut. H\"angen
\N{sre} und \N{maw} and \N{fwa} oder \N{krr}? Andere wahrscheinliche Kandidaten:
\N{fpi}, \N{m\`ikam}, \N{mungwrr}, \N{pxel/na}, \N{way}?}



\subsection{Substantivierungen als Konjunktionen} Im Na’vi gibt es ein paar
Konstruktionen im Zusammenhang mit Substantiven und dem attributiven Partikel, die
das tun, wof\"ur das Deutsche Konjunktionen verwendet. Daher haben scheinbar identische
Konjunktionen zwei unterschiedliche Formen --- eine, wenn die Konjunktion am Ende des
Gliedsatzes steht und eine, wenn sie an dessen Anfang steht. Oftmals sind diese
Satzglieder zu einem einzigen Wort verschmolzen, manchmal auch mit Lautverschiebungen.

\begin{center}
\begin{tabular}{rlll}
 & At the start & At the end \\
\hline
\D{nach} & \N{mawkrra} & \N{akrrmaw} & von \N{maw krr a} \\
\D{weil} & \N{talun(a)} & \N{alunta} & von \N{ta lun a} \\
\D{weil} & \N{taweyk(a)} & \QUAESTIO{\N{aweykta}} & von \N{ta oeyk a}\\
\D{wenn} & \N{krra} & \N{a krr} \\
\D{seit} (ab diesem Zeitpunkt) & \N{takrra} & \N{akrrta} & von \N{ta krr a}\\
\end{tabular}
\end{center}\label{syn:attr:takrra}
\index{mawkrra@\textbf{mawkrra}}\index{akrrmaw@\textbf{akrrmaw}}
\index{talun(a)@\textbf{talun(a)}}\index{alunta@\textbf{alunta}}
\index{taweyk(a)@\textbf{taweyk(a)}}\index{aweykta@\textbf{aweykta}}
\index{takrra@\textbf{takrra}}\index{akrrta@\textbf{akrrta}}
\index{krr@\textbf{krr}!mit attributivem \N{a}}
\NTeri{31/3/2012}{http://naviteri.org/2012/03/spring-vocabulary-part-2/}

\noindent \Npawl{T\`i’eyngit oel tolel \uwave{a krr}, ayngaru payeng},
\D{\uwave{Wenn} ich eine Antwort erhalte, werde ich es Dich wissen lassen} kann auch
zu \N{Krra t\`i’eyngit oel tolel, \dots} werden.
\LNWiki{1/2/2010}{http://wiki.learnnavi.org/index.php/Canon\%23Some_Conjunctions_and_Adverbs}
\LNWiki{1/2/2010}{http://wiki.learnnavi.org/index.php/Canon\%23Extracts_from_various_emails}
\NTeri{15/8/2011}{http://naviteri.org/2011/08/new-vocabulary-clothing/comment-page-1/\%23comment-986}



\section{Bedingte S\"atze}
\noindent Bedingte S\"atze werden im Na’vi mit der Konjunktion \N{txo}, \D{wenn}.
Die Konsequenzaussage wird wahlweise durch \N{tsakrr}, \N{dann}, eingeleitet.
\label{syn:conditionals}\index{Bedingter Satz}
\index{txo@\textbf{txo}}\index{tsakrr@\textbf{tsakrr}}

\subsection{Allgemein} \QUAESTIO{Bisher keine Beispiele.}
\index{Bedingter Satz!allgemein}

\subsection{Bedingte Zukunft} Im Deutschen verwenden bedingte Aussagen f\"ur die Bedingung
den Pr\"asens und das Futur I in der Konsequenzaussage: "`Wenn Du dieses machst, werde ich
jenes machen."' Im Na’vi steht die Bedingung im Subjunktiv und die Konsequenz im Futur:
\Npawl{Pxan \uwave{l\INF{iv}u} txo n\`i’aw oe ngari / Tsakrr nga Na’viru
\uwave{yomt\INF{\`iy}\`ing}} \D{Nur wenn ich Deiner w\"urdig \uwave{bin}, \uwave{wirst} Du
das Volk versorgen}.\index{Bedingter Satz!Futur}

\subsection{Hypothetisch} \QUAESTIO{Bisher keine Beispiele.}
\index{Bedingter Satz!hypothetisch}

\subsection{Widerspruch} \QUAESTIO{Bisher keine Beispiele.}
\index{Bedingter Satz!widerspr\"uchlich}

\subsection{Imperative in Bedingungen} Werden Imperative in der Konsequenzaussage verwendet,
so haben imperative Stimmungslage und Satzbauregeln Vorrang vor den normalen Mustern einer
Bedingung. Zum Beispiel eine bedingte Zukunft mit einem Imperativ als Konsequenz:
\Npawl{Txo \uwave{tsive’a} ayngal keyeyt, rutxe oeru \uwave{piveng}},
\D{Wenn Sie Fehler finden, geben Sie mir bitte Bescheid}.
\index{Imperativ!in bedingten S\"atzen}


\section{Konjunktionen}
\noindent Dieser Abschnitt z\"ahlt Konjunktionen auf, die sonst nirgendwo diskutiert worden
sind, aber die dennoch erw\"ahnt werden sollten. Ich werde Konjunktionen, die keiner
gesonderten Kommentierung bed\"urfen, auslassen.

\subsection{Alu} Die Hauptanwendung von \N{alu} ist mit Substantiven in Apposition:
\Npawl{Tskalepit oel tol\`ing oey\"a \uwave{tsmukanur alu \`Istaw}},
\D{Ich gab \uwave{meinem Bruder \`Istaw} den Bogen}. Beachten Sie, da\ss{} das Substantiv
hinter \N{alu} im Nominativ steht.
\NTeri{16/7/2010}{http://naviteri.org/2010/07/vocabulary-update/}
\index{alu@\textbf{alu}}\label{syn:conj:alu}\index{Apposition}

\subsubsection{} \N{Alu} kann auch gespr\"achsbezogen eingesetzt werden, um eine
Aussagewiederholung zu kennzeichnen, so wie "`das soll hei\ss{}en"' oder "`in anderen
Worten"'.
\Npawl{Txoa livu, yawne lu oer Sorewn...\ \uwave{alu}...\ ke tsun oeng muntxa slivu},
\D{Verzeihung, aber ich liebe Sorewn... \uwave{soll hei\ss{}en}... wir k\"onnen nicht
heiraten}.

\subsubsection{} In Diskussionen bez\"uglich Grammatik und Sprache kann \N{alu} die Konstruktion,
\"uber die gesprochen wird, verdeutlichen: \Npawl{Tsalsungay tsal\`i’u \uwave{alu zeykuso} lu
eyawr}, \D{Dennoch war das Wort \uwave{zeykuso} richtig} oder \Npawl{L\`i’uri alu tskxe pamrel
fyape}, \D{Wie buchstabiert man das Wort '`tskxe'’}?

\subsection{Ftxey} Zus\"atzlich zum Erzeugen von Ja-Nein-Fragen (\horenref{syn:question:ftxey})
kann \N{ftxey} verwendet werden, um M\"oglichkeiten aufzuz\"ahlen: \Npawl{S\`ilpey oe, \dots\
frapo — \uwave{ftxey sng\"a’iyu ftxey tsulf\"atu} — ts\`iyevun f\`itsenge rivun ’uot
lesar}, \D{Ich hoffe ... jeder — \uwave{ob Anf\"anger oder Experte} — wird in der Lage sein,
etwas N\"utzliches zu finden}.
\index{ftxey@\textbf{ftxey}}

\subsection{Fu} Die Konjunktion \N{fu}, \D{oder}, kann verwendet werden, um Nominalphrasen
und Verbkonstruktionen miteinander zu verbinden.
\Npawl{Ke zasyup l\`i’Ona ne kxutu a m\`ifa fu a wrrpa}, \D{Die L\`i’Ona werden weder dem
\"au\ss{}eren noch dem inneren Feind anheimfallen}. \QUAESTIO{Bisher noch keine Beispiele f\"ur
Verbkonstruktionen...}\index{fu@\textbf{fu}}

\subsection{Ki} Die Konjunktion \N{ki}, \D{sondern, stattdessen}, wird mit dem negativen
Adverb \N{ke} zusammen verwendet. Achten Sie darauf, es nicht mit \N{sl\"a}, \D{aber}, zu
verwechseln: \Npawl{Nga plltxe ke n\`ifyeyntu ki n\`i’eveng}, \D{Du sprichst nicht wie ein
Erwachsener, sondern wie ein Kind}.
\NTeri{16/7/2010}{http://naviteri.org/2010/07/vocabulary-update/}
\index{ki@\textbf{ki}}

\subsection{S\`i} Die Konjunktion \N{s\`i}, \D{und}, wird verwendet, um Listen zu erstellen
und Elemente gleichen Konzeptes zu kombinieren. Sie wird nicht verwendet, um Glieds\"atze
miteinander zu verbinden, was stattdessen die Aufgabe von \N{ulte} ist (\horenref{syn:ulte}).
\Npawl{Lu p\`ilokur pxes\`ikan s\`i pxefne’upxare}, \D{Der Blog umfa\ss{}t drei Anforderungen
und drei Arten Nachrichten} oder \Npawl{Ma smukan s\`i smuke}, \D{Br\"uder und Schwestern}.
\index{siì@\textbf{s\`i}}\label{syn:sì}

\subsubsection{} Obwohl \N{s\`i} am h\"aufigsten daf\"ur verwendet wird, um Nominalphrasen,
Pronomen und Adjektive miteinander zu verbinden, kann es auch eng miteinander in Beziehung
stehende Verben zusammenzufassen: \Nfilm{S\"anume sivi poru fte \uwave{pivlltxe s\`i
tiv\`iran} n\`iayoeng}, \D{Lehre ihn, wie wir \uwave{zu reden und zu gehen}}.

\subsubsection{} \N{S\`i} kann auch enkliktisch sein (\horenref{l-and-s:stress:enclisis}).
Hierbei folgt es dem Wort oder Satzglied, das es der Liste hinzuf\"ugt: \Npawl{Ta ’eylan
\uwave{karyus\`i} ayngey\"a, Pawl}, \D{Von Deinem \uwave{Freud und Lehrer} Paul} oder
\Npawl{Tsakrr paye’un sweya fya’ot a zamivunge oel ayngar ayl\`i’ut \uwave{horentis\`i}
l\`i’fyayä leNa’vi}, \D{Und dann werde ich entscheiden, welches der beste Weg ist, euch
die W\"orter \uwave{und Regeln} des Na’vi zu \"uberbringen}.\index{siì@\textbf{s\`i}!enkliktisch}

%\subsection{Tengfya} \QUAESTIO{Needed?}

\subsection{Tengkrr} Der Sinn von \N{tengkrr}, \D{w\"ahrend, zur gleichen Zeit wie,
gleichzeitig} verlangt, da\ss{} es zusammen mit dem Imperfektiv verwendet wird:
\Npawl{F\`itxon yom \uwave{tengkrr teruvon}}, \D{Diese Nacht lernen wir, w\"ahrend wir
essen}.\index{tengkrr@\textbf{tengkrr}}
\LNWiki{14/3/2010}{http://wiki.learnnavi.org/index.php/Canon/2010/March-June\%23A_Collection}

\subsection{Tsn\`i} Die Konjunktion \N{tsn\`i}, \D{da\ss{}}, leitet einige Arten von
Aussages\"atzen ein, in denen das Verb im Subjunktiv steht: \N{\"Atx\"ale si tsn\`i livu oheru
Uniltaron}, \D{Ich erbitte h\"oflich die Traumjagd} oder \Npawl{S\`ilpey oe tsn\`i
f\`it\`ioeykt\`ing law livu ngaru set}, \D{Ich hoffe, da\ss{} diese Erkl\"arung Dir jetzt
klar wird}.
\QUAESTIO{Bestimmt das Verb oder die Konstruktion den Subjunktiv?}
% http://naviteri.org/2011/02/new-vocabulary-part-2/
\label{syn:tsni}\index{tsnì@\textbf{tsn\`i}}

\subsubsection{} \QUAESTIO{\N{Tsn\`i} scheint am h\"aufigsten zusammen mit intransitiven
Konstruktionen im Hauptsatz verwendet zu werden}.

%\subsubsection{} Verbs known to take \N{tsnì}: \N{ätxäle si},
%\N{rangal} (a marginal use), \N{sìlpey}.
% http://forum.learnnavi.org/language-updates/confirmation-on-use-of-rangal/

\subsection{Ulte} Diese Konjunktion verbindet Glieds\"atze miteinander: \N{Oel ngati
kameie, ma tsmukan, \uwave{ulte} ngaru seiyi irayo}, \D{Ich Sehe Dich, Bruder,
\uwave{und} ich danke Dir}. Bitte nicht mit \N{s\`i} verwechseln.
(\horenref{syn:sì}).
\index{ulte@\textbf{ulte}}\label{syn:ulte}



\section{Direkte Rede}
\label{syn:direct-quote}

\subsection{San... s\`ik} Das Na’vi verf\"ugt \"uber keine indirekte Rede (\D{Er sagte, da\ss{}
sie gegangen w\"aren}), sondern verwendet die direkte Rede, wobei die zitierten Worte zwischen
die Partikel \N{san} und \N{s\`ik} stehen, wie in \Npawl{Sl\"a n\`i’i’a tsun oe pivlltxe
\uwave{san Zola’u n\`iprrte’ ne p\`ilok Na’viteri s\`ik}}, \D{Aber jetzt kann ich endlich sagen:
"`\uwave{Willkommen auf dem Blog Na’viteri}"'!}\index{W\"ortliche Rede!direkt}
\index{Zitat!direkt}
\index{san@\textbf{san}}\index{siìk@\textbf{s\`ik}}
\NTeri{31/8/2011}{http://naviteri.org/2011/08/reported-speech-reported-questions/}

\subsubsection{} F\"allt der Anfang oder das Ende der w\"ortlichen Rede mit dem Anfang oder
Ende eines Ausrufs, so kann ein Teil des Partikelpaars \N{san \dots s\`ik} fallengelassen
werden.

\begin{quotation}
\noindent 1. \Npawl{Poltxe Eytukan \uwave{san} oe kay\"a \uwave{s\`ik}, sl\"a oel pot ke
spaw.}\\
\indent\D{Eytukan sagte, da\ss{} er gehen wird, aber ich glaube ihm nicht}.

\noindent 2. \N{Poltxe Eytukan \uwave{san} oe kay\"a.}\\
\indent\D{Eytukan sagte, da\ss{} er gehen wird.}
\end{quotation}

\noindent Da in 2. nichts hinter der direkten Rede folgt, kann das abschlie\ss{}ende \N{s\`ik}
entfallen.
\LNWiki{21/1/2010}{http://wiki.learnnavi.org/index.php/Canon\%23Extracts_from_various_emails}

\subsection{Fragen} Indirekte Fragen werden ebenfalls direkt zitiert: \Npawl{Polawm po san srake
S\"ali holum s\`ik}, \D{Sie fragte, ob Sally gegangen ist}, w\"ortlich: \D{Er fragte: "`Ist Sally
gegangen?"'}
\LNWiki{24/3/2010}{http://wiki.learnnavi.org/index.php/Canon/2010/March-June\%23If_and_Whether}

\subsubsection{} Mit \N{pawm}, aber nicht mit anderen Verben des Sprechens, kann \N{san \dots
s\`ik} entfallen: \Npawl{Polawm po, Neytiri k\"a pesengne,}, \D{Er fragte, wohin Neytiri ging.}
\NTeri{31/8/2011}{http://naviteri.org/2011/08/reported-speech-reported-questions/}


\subsection{Transitivit\"at} Wenn ein Verb des Sprechens \N{san \dots s\`ik} verwendet, so
verwendet es eine intransitive Satzkonstruktion: \N{Po poltxe san srane}, \D{Er sagte: "`Ja."'}
\index{Transitivit\"at!mit Verben des Sprechens}
\Ultxa{2/10/2010}{http://wiki.learnnavi.org/index.php/Canon/2010/UltxaAyharyu\%C3\%A4\%23Transitivity_with_Speaking_Verbs}

\subsubsection{} Hat das Verb des Sprechens ein direktes Objekt, so verwendet es eine
transitive Satzkonstruktion: \Npawl{Ke poltxe pol tsayl\`i’ut}, \D{Sie hat das nicht gesagt}:
oder \N{Oel poru pasyawn tsat}, \D{Ich werde ihn das fragen.}
\NTeri{31/8/2011}{http://naviteri.org/2011/08/reported-speech-reported-questions/}

\subsection{Zitatsubstantivierung} Zus\"atzlich zu \N{san \dots s\`ik} kann die indirekte
Rede auch mittels des attributiven \N{a} (siehe auch \horenref{morph:fmawn} f\"ur Verschmelzungen)
an die W\"orter \N{fmawn}, \D{Nachricht}, \N{t\`i’eyng}, \D{Antwort} und \N{fayl\`i’u}, \D{diese
Worte} geh\"angt werden.
\NTeri{31/8/2011}{http://naviteri.org/2011/08/reported-speech-reported-questions/}

\begin{center}
\begin{tabular}{ll}
Verb & Zitierung \\
\hline
\N{plltxe}, \D{sagen} & \N{san... s\`ik}, \N{fayl\`i’u} \\
\N{stawm}, \D{h\"oren}, \N{peng}, \D{mitteilen} & \N{fmawn} \\
\N{pawm}, \D{fragen} & \N{san... s\`ik}, \N{t\`i’eyng}, nichts \\
\N{vin} \D{fragen (nach)} & \N{t\`i’eyng} 
\end{tabular}
\end{center}
\index{pawm@\textbf{pawm}}\index{stawm@\textbf{stawm}}
\index{plltxe@\textbf{plltxe}}\index{vin@\textbf{vin}}

\noindent Zitierungen, die an diese W\"orter geh\"angt werden, stehen immer noch in der
direkten Rede.

\begin{quotation}
\noindent\Npawl{Poltxe pol fayluta oe new kiv\"a}, \D{Sie sagte, sie wollte gehen}. \\
\indent w\"ortl., \D{Sie sagte: "`Ich will gehen."'} \\
\noindent\Npawl{Ngal poleng oer fmawnta po tolerkup}.
\indent\D{Du sagtest mir, da\ss{} sie gestorben w\"are}. \\
\noindent\N{Volin pol t\`i’eyngit a Neytiri k\"a pesengne}. \\
\indent\D{Er fragte, wohin Neystiri ging}.
\end{quotation}
\label{syn:quot:nominalized}

\subsubsection{} Andere Verben, die von indirekter Rede verwendet werden k\"onnen,
k\"onnen Substantivierungen mit \N{t\`i’eyng} verwenden.

\begin{quotation}
\noindent\Npawl{Ke omum oel teyngta fo k\"a pesengne}.\\
\indent\D{Ich wei\ss{} nicht, wohin sie gehen}. \\
\noindent\Npawl{Teynga lumpe fo holum ke lu law}. \\
\indent\D{Es ist nicht klar, weshalb sie gegangen sind}.
\end{quotation}


\section{Partikel}

\subsection{Ko} Das Satzendepartikel \N{ko} wird verwendet, um ein Einverst\"andnis
verschiedenster Art einzuholen, einschlie\ss{}lich solcher Bedeutungen wie "`La\ss{}t
uns\dots"', "`meinst Du nicht"', "`Warum machst Du nicht"', "`Warum mache ich nicht"'.
Oftmals ist im Film \D{Makto ko}, \D{La\ss{}t uns losreiten} oder \N{Kiv\"a ko},
\D{La\ss{} uns gehen} zu h\"oren.\index{ko@\textbf{ko}}\label{syn:particle:ko}

\subsection{Nang} Dieses Partikel zeigt \"Uberraschung, einen Ausruf oder Ermutigung an.
Es steht immer am Ende des Satzes und erscheint zusammen mit Adverben des Ma\ss{}es oder
der Zustimmung, so wie in \N{n\`ingay}, \N{n\`itxan}, \N{f\`itxan}, usw.
\N{Txantsana s\`ipawm apxay f\`itxan lu ngaru nang!} \D{Du hast so viele ausgezeichnete
Fragen!}
\Npawl{Ngari tswintsy\`ip sevin n\`itxan lu nang!} \D{Welch h\"ubschen, kleinen Tswin
Du doch hast!}
\index{nang@\textbf{nang}}

\subsection{Pak} Dieses Partikel folgt dem Wort, zu dem es geh\"ort und dr\"uckt
Herabsetzung oder Verunglimpfung aus: \N{Tsamsiyu pak!} \D{Ein Krieger? Schon klar!}.
\index{pak@\textbf{pak}}

\subsection{Tut} Dies ist ein Fortsetzungspartikel, das bisher nur in Fortsetzungen einer
Frage in Dialogen in Erscheinung tritt.

\begin{quotation}
\noindent A: \N{Ngaru lu fpom srak?} \D{Wie geht es Dir?} \\
\noindent B: \N{Oeru lu fpom. \uwave{Ngaru tut?}} \D{Mir geht es gut. \uwave{Und Dir?}}
\end{quotation}
\index{tut@\textbf{tut}}

\subsection{Tse} Dieses Partikel zeigt ein Z\"ogern in einer Unterhaltung an:
\D{Also, nun ja}.
\index{tse@\textbf{tse}}


\section{Andere erw\"ahnenswerte W\"orter}

\subsection{Sweylu}\label{syn:sweylu} Die Syntax dieses Verbs, das \D{sollte} bedeutet
(abgeleitet von \N{swey lu}, \D{es ist am besten}) \"andert sich abh\"angig davon, ob
die Verpflichtung sich auf etwas bezieht, das noch nicht geschehen ist, oder auf etwas,
das bereits stattgefunden hat.
\index{sweylu@\textbf{sweylu}}

\subsubsection{} Damit es sich auf die Zukunft bezieht, wird \N{txo} zusammen mit dem
Subjunktiv verwendet: \Npawl{Sweylu txo \uwave{nga} kiv\"a} oder \N{\uwave{Nga} sweylu
txo kiv\"a} f\"ur \D{\uwave{Du} solltest gehen}. Beachten Sie, da\ss{} die Verneinung
im \N{txo}-Satz steht: \Npawl{Sweylu txo nga ke kiv\"a} oder \N{Mga sweylu txo ke kiv\"a},
\D{Du solltest nicht gehen}.

\subsubsection{} F\"ur etwas, das bereits passiert ist, verwenden Sie \N{fwa} oder \N{tsawa}
zusammen mit dem Indikativ Pr\"ateritum oder Perfekt.

\begin{quotation}
\noindent Tsenu: \Npawl{Spaw oe, fwa po kol\"a l\"angu kxeyey.}\\
\noindent\D{Ich glaube, es war ein Fehler von ihm zu gehen.} \\

\noindent Kamun: \N{Kehe, kehe! Sweylu fwa po kol\"a!}\\
\noindent\D{Nein, nein! Er h\"atte nicht gehen sollen!}
\end{quotation}

\noindent Beachten Sie, da\ss{} sich dies auf ein Ereignis in der Vergangenheit, das zu tun
nicht das Richtige gewesen war, und nicht auf eine nicht erf\"ullte vergangene Handlung (was
eine andere Anwendung des deutschen "`sollte"' ist).
